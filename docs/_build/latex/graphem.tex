%% Generated by Sphinx.
\def\sphinxdocclass{report}
\documentclass[letterpaper,10pt,english]{sphinxmanual}
\ifdefined\pdfpxdimen
   \let\sphinxpxdimen\pdfpxdimen\else\newdimen\sphinxpxdimen
\fi \sphinxpxdimen=.75bp\relax
\ifdefined\pdfimageresolution
    \pdfimageresolution= \numexpr \dimexpr1in\relax/\sphinxpxdimen\relax
\fi
%% let collapsible pdf bookmarks panel have high depth per default
\PassOptionsToPackage{bookmarksdepth=5}{hyperref}

\PassOptionsToPackage{booktabs}{sphinx}
\PassOptionsToPackage{colorrows}{sphinx}

\PassOptionsToPackage{warn}{textcomp}
\usepackage[utf8]{inputenc}
\ifdefined\DeclareUnicodeCharacter
% support both utf8 and utf8x syntaxes
  \ifdefined\DeclareUnicodeCharacterAsOptional
    \def\sphinxDUC#1{\DeclareUnicodeCharacter{"#1}}
  \else
    \let\sphinxDUC\DeclareUnicodeCharacter
  \fi
  \sphinxDUC{00A0}{\nobreakspace}
  \sphinxDUC{2500}{\sphinxunichar{2500}}
  \sphinxDUC{2502}{\sphinxunichar{2502}}
  \sphinxDUC{2514}{\sphinxunichar{2514}}
  \sphinxDUC{251C}{\sphinxunichar{251C}}
  \sphinxDUC{2572}{\textbackslash}
\fi
\usepackage{cmap}
\usepackage[T1]{fontenc}
\usepackage{amsmath,amssymb,amstext}
\usepackage{babel}



\usepackage{tgtermes}
\usepackage{tgheros}
\renewcommand{\ttdefault}{txtt}



\usepackage[Bjarne]{fncychap}
\usepackage{sphinx}

\fvset{fontsize=auto}
\usepackage{geometry}


% Include hyperref last.
\usepackage{hyperref}
% Fix anchor placement for figures with captions.
\usepackage{hypcap}% it must be loaded after hyperref.
% Set up styles of URL: it should be placed after hyperref.
\urlstyle{same}

\addto\captionsenglish{\renewcommand{\contentsname}{Contents:}}

\usepackage{sphinxmessages}
\setcounter{tocdepth}{1}



\title{GraphEm}
\date{May 30, 2025}
\release{0.1.0}
\author{Alexander Kolpakov (UATX), Igor Rivin (Temple University)}
\newcommand{\sphinxlogo}{\vbox{}}
\renewcommand{\releasename}{Release}
\makeindex
\begin{document}

\ifdefined\shorthandoff
  \ifnum\catcode`\=\string=\active\shorthandoff{=}\fi
  \ifnum\catcode`\"=\active\shorthandoff{"}\fi
\fi

\pagestyle{empty}
\sphinxmaketitle
\pagestyle{plain}
\sphinxtableofcontents
\pagestyle{normal}
\phantomsection\label{\detokenize{index::doc}}


\sphinxstepscope


\chapter{graphem}
\label{\detokenize{modules:graphem}}\label{\detokenize{modules::doc}}
\sphinxstepscope


\section{graphem package}
\label{\detokenize{graphem:graphem-package}}\label{\detokenize{graphem::doc}}

\subsection{Submodules}
\label{\detokenize{graphem:submodules}}

\subsection{graphem.benchmark module}
\label{\detokenize{graphem:module-graphem.benchmark}}\label{\detokenize{graphem:graphem-benchmark-module}}\index{module@\spxentry{module}!graphem.benchmark@\spxentry{graphem.benchmark}}\index{graphem.benchmark@\spxentry{graphem.benchmark}!module@\spxentry{module}}
\sphinxAtStartPar
Benchmark functionality for Graphem.
\index{run\_benchmark() (in module graphem.benchmark)@\spxentry{run\_benchmark()}\spxextra{in module graphem.benchmark}}

\begin{fulllineitems}
\phantomsection\label{\detokenize{graphem:graphem.benchmark.run_benchmark}}
\pysigstartsignatures
\pysiglinewithargsret{\sphinxcode{\sphinxupquote{graphem.benchmark.}}\sphinxbfcode{\sphinxupquote{run\_benchmark}}}{\sphinxparam{\DUrole{n,n}{graph\_generator}}\sphinxparamcomma \sphinxparam{\DUrole{n,n}{graph\_params}}\sphinxparamcomma \sphinxparam{\DUrole{n,n}{dim}\DUrole{o,o}{=}\DUrole{default_value}{3}}\sphinxparamcomma \sphinxparam{\DUrole{n,n}{L\_min}\DUrole{o,o}{=}\DUrole{default_value}{10.0}}\sphinxparamcomma \sphinxparam{\DUrole{n,n}{k\_attr}\DUrole{o,o}{=}\DUrole{default_value}{0.5}}\sphinxparamcomma \sphinxparam{\DUrole{n,n}{k\_inter}\DUrole{o,o}{=}\DUrole{default_value}{0.1}}\sphinxparamcomma \sphinxparam{\DUrole{n,n}{knn\_k}\DUrole{o,o}{=}\DUrole{default_value}{15}}\sphinxparamcomma \sphinxparam{\DUrole{n,n}{sample\_size}\DUrole{o,o}{=}\DUrole{default_value}{512}}\sphinxparamcomma \sphinxparam{\DUrole{n,n}{batch\_size}\DUrole{o,o}{=}\DUrole{default_value}{1024}}\sphinxparamcomma \sphinxparam{\DUrole{n,n}{num\_iterations}\DUrole{o,o}{=}\DUrole{default_value}{40}}}{}
\pysigstopsignatures
\sphinxAtStartPar
Run a benchmark on the given graph.
\begin{quote}\begin{description}
\sphinxlineitem{Parameters}\begin{itemize}
\item {} 
\sphinxAtStartPar
\sphinxstyleliteralstrong{\sphinxupquote{graph\_generator}} \textendash{} callable
Function to generate a graph

\item {} 
\sphinxAtStartPar
\sphinxstyleliteralstrong{\sphinxupquote{graph\_params}} \textendash{} dict
Parameters for the graph generator

\item {} 
\sphinxAtStartPar
\sphinxstyleliteralstrong{\sphinxupquote{dim}} \textendash{} int
Embedding dimension

\item {} 
\sphinxAtStartPar
\sphinxstyleliteralstrong{\sphinxupquote{L\_min}} \textendash{} float
GraphEmbedder parameters

\item {} 
\sphinxAtStartPar
\sphinxstyleliteralstrong{\sphinxupquote{k\_attr}} \textendash{} float
GraphEmbedder parameters

\item {} 
\sphinxAtStartPar
\sphinxstyleliteralstrong{\sphinxupquote{k\_inter}} \textendash{} float
GraphEmbedder parameters

\item {} 
\sphinxAtStartPar
\sphinxstyleliteralstrong{\sphinxupquote{knn\_k}} \textendash{} float
GraphEmbedder parameters

\item {} 
\sphinxAtStartPar
\sphinxstyleliteralstrong{\sphinxupquote{sample\_size}} \textendash{} int
Batch parameters for kNN search

\item {} 
\sphinxAtStartPar
\sphinxstyleliteralstrong{\sphinxupquote{batch\_size}} \textendash{} int
Batch parameters for kNN search

\item {} 
\sphinxAtStartPar
\sphinxstyleliteralstrong{\sphinxupquote{num\_iterations}} \textendash{} int
Number of layout iterations

\end{itemize}

\sphinxlineitem{Returns}
\sphinxAtStartPar
Benchmark results including timings and graph metrics

\sphinxlineitem{Return type}
\sphinxAtStartPar
\sphinxhref{https://docs.python.org/3/library/stdtypes.html\#dict}{dict}

\end{description}\end{quote}

\end{fulllineitems}

\index{benchmark\_correlations() (in module graphem.benchmark)@\spxentry{benchmark\_correlations()}\spxextra{in module graphem.benchmark}}

\begin{fulllineitems}
\phantomsection\label{\detokenize{graphem:graphem.benchmark.benchmark_correlations}}
\pysigstartsignatures
\pysiglinewithargsret{\sphinxcode{\sphinxupquote{graphem.benchmark.}}\sphinxbfcode{\sphinxupquote{benchmark\_correlations}}}{\sphinxparam{\DUrole{n,n}{graph\_generator}}\sphinxparamcomma \sphinxparam{\DUrole{n,n}{graph\_params}}\sphinxparamcomma \sphinxparam{\DUrole{n,n}{dim}\DUrole{o,o}{=}\DUrole{default_value}{2}}\sphinxparamcomma \sphinxparam{\DUrole{n,n}{L\_min}\DUrole{o,o}{=}\DUrole{default_value}{10.0}}\sphinxparamcomma \sphinxparam{\DUrole{n,n}{k\_attr}\DUrole{o,o}{=}\DUrole{default_value}{0.5}}\sphinxparamcomma \sphinxparam{\DUrole{n,n}{k\_inter}\DUrole{o,o}{=}\DUrole{default_value}{0.1}}\sphinxparamcomma \sphinxparam{\DUrole{n,n}{knn\_k}\DUrole{o,o}{=}\DUrole{default_value}{15}}\sphinxparamcomma \sphinxparam{\DUrole{n,n}{sample\_size}\DUrole{o,o}{=}\DUrole{default_value}{512}}\sphinxparamcomma \sphinxparam{\DUrole{n,n}{batch\_size}\DUrole{o,o}{=}\DUrole{default_value}{1024}}\sphinxparamcomma \sphinxparam{\DUrole{n,n}{num\_iterations}\DUrole{o,o}{=}\DUrole{default_value}{40}}}{}
\pysigstopsignatures
\sphinxAtStartPar
Run a benchmark to calculate correlations between embedding radii and centrality measures.
\begin{quote}\begin{description}
\sphinxlineitem{Parameters}\begin{itemize}
\item {} 
\sphinxAtStartPar
\sphinxstyleliteralstrong{\sphinxupquote{graph\_generator}} \textendash{} callable
Function to generate a graph

\item {} 
\sphinxAtStartPar
\sphinxstyleliteralstrong{\sphinxupquote{graph\_params}} \textendash{} dict
Parameters for the graph generator

\item {} 
\sphinxAtStartPar
\sphinxstyleliteralstrong{\sphinxupquote{parameters}} (\sphinxstyleliteralemphasis{\sphinxupquote{Other}}) \textendash{} same as run\_benchmark

\end{itemize}

\sphinxlineitem{Returns}
\sphinxAtStartPar
Benchmark results with correlations

\sphinxlineitem{Return type}
\sphinxAtStartPar
\sphinxhref{https://docs.python.org/3/library/stdtypes.html\#dict}{dict}

\end{description}\end{quote}

\end{fulllineitems}

\index{run\_influence\_benchmark() (in module graphem.benchmark)@\spxentry{run\_influence\_benchmark()}\spxextra{in module graphem.benchmark}}

\begin{fulllineitems}
\phantomsection\label{\detokenize{graphem:graphem.benchmark.run_influence_benchmark}}
\pysigstartsignatures
\pysiglinewithargsret{\sphinxcode{\sphinxupquote{graphem.benchmark.}}\sphinxbfcode{\sphinxupquote{run\_influence\_benchmark}}}{\sphinxparam{\DUrole{n,n}{graph\_generator}}\sphinxparamcomma \sphinxparam{\DUrole{n,n}{graph\_params}}\sphinxparamcomma \sphinxparam{\DUrole{n,n}{k}\DUrole{o,o}{=}\DUrole{default_value}{10}}\sphinxparamcomma \sphinxparam{\DUrole{n,n}{p}\DUrole{o,o}{=}\DUrole{default_value}{0.1}}\sphinxparamcomma \sphinxparam{\DUrole{n,n}{iterations}\DUrole{o,o}{=}\DUrole{default_value}{200}}\sphinxparamcomma \sphinxparam{\DUrole{n,n}{dim}\DUrole{o,o}{=}\DUrole{default_value}{3}}\sphinxparamcomma \sphinxparam{\DUrole{n,n}{num\_layout\_iterations}\DUrole{o,o}{=}\DUrole{default_value}{20}}\sphinxparamcomma \sphinxparam{\DUrole{n,n}{layout\_params}\DUrole{o,o}{=}\DUrole{default_value}{None}}}{}
\pysigstopsignatures
\sphinxAtStartPar
Run a benchmark comparing influence maximization methods.
\begin{quote}\begin{description}
\sphinxlineitem{Parameters}\begin{itemize}
\item {} 
\sphinxAtStartPar
\sphinxstyleliteralstrong{\sphinxupquote{graph\_generator}} \textendash{} callable
Function to generate a graph

\item {} 
\sphinxAtStartPar
\sphinxstyleliteralstrong{\sphinxupquote{graph\_params}} \textendash{} dict
Parameters for the graph generator

\item {} 
\sphinxAtStartPar
\sphinxstyleliteralstrong{\sphinxupquote{k}} \textendash{} int
Number of seed nodes to select

\item {} 
\sphinxAtStartPar
\sphinxstyleliteralstrong{\sphinxupquote{p}} \textendash{} float
Propagation probability

\item {} 
\sphinxAtStartPar
\sphinxstyleliteralstrong{\sphinxupquote{iterations}} \textendash{} int
Number of iterations for influence simulation

\item {} 
\sphinxAtStartPar
\sphinxstyleliteralstrong{\sphinxupquote{dim}} \textendash{} int
Embedding dimension

\item {} 
\sphinxAtStartPar
\sphinxstyleliteralstrong{\sphinxupquote{num\_layout\_iterations}} \textendash{} int
Number of iterations for layout algorithm

\item {} 
\sphinxAtStartPar
\sphinxstyleliteralstrong{\sphinxupquote{layout\_params}} \textendash{} dict
Parameters for the layout algorithm

\end{itemize}

\sphinxlineitem{Returns}
\sphinxAtStartPar
Benchmark results comparing influence maximization methods

\sphinxlineitem{Return type}
\sphinxAtStartPar
\sphinxhref{https://docs.python.org/3/library/stdtypes.html\#dict}{dict}

\end{description}\end{quote}

\end{fulllineitems}



\subsection{graphem.datasets module}
\label{\detokenize{graphem:module-graphem.datasets}}\label{\detokenize{graphem:graphem-datasets-module}}\index{module@\spxentry{module}!graphem.datasets@\spxentry{graphem.datasets}}\index{graphem.datasets@\spxentry{graphem.datasets}!module@\spxentry{module}}
\sphinxAtStartPar
Real\sphinxhyphen{}world dataset loader for Graphem.

\sphinxAtStartPar
This module provides functions to download and load standard large graph datasets
from various sources including SNAP (Stanford Network Analysis Project),
Network Repository, and other public graph repositories.
\index{get\_data\_directory() (in module graphem.datasets)@\spxentry{get\_data\_directory()}\spxextra{in module graphem.datasets}}

\begin{fulllineitems}
\phantomsection\label{\detokenize{graphem:graphem.datasets.get_data_directory}}
\pysigstartsignatures
\pysiglinewithargsret{\sphinxcode{\sphinxupquote{graphem.datasets.}}\sphinxbfcode{\sphinxupquote{get\_data\_directory}}}{}{}
\pysigstopsignatures
\sphinxAtStartPar
Get the data directory for storing downloaded datasets.
Creates the directory if it doesn’t exist.
\begin{quote}\begin{description}
\sphinxlineitem{Returns}
\sphinxAtStartPar
Path to the data directory

\sphinxlineitem{Return type}
\sphinxAtStartPar
Path

\end{description}\end{quote}

\end{fulllineitems}

\index{download\_file() (in module graphem.datasets)@\spxentry{download\_file()}\spxextra{in module graphem.datasets}}

\begin{fulllineitems}
\phantomsection\label{\detokenize{graphem:graphem.datasets.download_file}}
\pysigstartsignatures
\pysiglinewithargsret{\sphinxcode{\sphinxupquote{graphem.datasets.}}\sphinxbfcode{\sphinxupquote{download\_file}}}{\sphinxparam{\DUrole{n,n}{url}}\sphinxparamcomma \sphinxparam{\DUrole{n,n}{filepath}}\sphinxparamcomma \sphinxparam{\DUrole{n,n}{description}\DUrole{o,o}{=}\DUrole{default_value}{None}}}{}
\pysigstopsignatures
\sphinxAtStartPar
Download a file with progress bar.
\begin{quote}\begin{description}
\sphinxlineitem{Parameters}\begin{itemize}
\item {} 
\sphinxAtStartPar
\sphinxstyleliteralstrong{\sphinxupquote{url}} \textendash{} str
URL to download

\item {} 
\sphinxAtStartPar
\sphinxstyleliteralstrong{\sphinxupquote{filepath}} \textendash{} Path or str
Path to save the downloaded file

\item {} 
\sphinxAtStartPar
\sphinxstyleliteralstrong{\sphinxupquote{description}} \textendash{} str, optional
Description for the progress bar

\end{itemize}

\end{description}\end{quote}

\end{fulllineitems}

\index{extract\_file() (in module graphem.datasets)@\spxentry{extract\_file()}\spxextra{in module graphem.datasets}}

\begin{fulllineitems}
\phantomsection\label{\detokenize{graphem:graphem.datasets.extract_file}}
\pysigstartsignatures
\pysiglinewithargsret{\sphinxcode{\sphinxupquote{graphem.datasets.}}\sphinxbfcode{\sphinxupquote{extract\_file}}}{\sphinxparam{\DUrole{n,n}{filepath}}\sphinxparamcomma \sphinxparam{\DUrole{n,n}{extract\_dir}\DUrole{o,o}{=}\DUrole{default_value}{None}}}{}
\pysigstopsignatures
\sphinxAtStartPar
Extract a compressed file.
\begin{quote}\begin{description}
\sphinxlineitem{Parameters}\begin{itemize}
\item {} 
\sphinxAtStartPar
\sphinxstyleliteralstrong{\sphinxupquote{filepath}} \textendash{} Path or str
Path to the compressed file

\item {} 
\sphinxAtStartPar
\sphinxstyleliteralstrong{\sphinxupquote{extract\_dir}} \textendash{} Path or str, optional
Directory to extract to. If None, extracts to the same directory as the file.

\end{itemize}

\sphinxlineitem{Returns}
\sphinxAtStartPar
Path to the extraction directory

\sphinxlineitem{Return type}
\sphinxAtStartPar
Path

\end{description}\end{quote}

\end{fulllineitems}

\index{DatasetLoader (class in graphem.datasets)@\spxentry{DatasetLoader}\spxextra{class in graphem.datasets}}

\begin{fulllineitems}
\phantomsection\label{\detokenize{graphem:graphem.datasets.DatasetLoader}}
\pysigstartsignatures
\pysiglinewithargsret{\sphinxbfcode{\sphinxupquote{class\DUrole{w,w}{  }}}\sphinxcode{\sphinxupquote{graphem.datasets.}}\sphinxbfcode{\sphinxupquote{DatasetLoader}}}{\sphinxparam{\DUrole{n,n}{name}}}{}
\pysigstopsignatures
\sphinxAtStartPar
Bases: \sphinxhref{https://docs.python.org/3/library/functions.html\#object}{\sphinxcode{\sphinxupquote{object}}}

\sphinxAtStartPar
Base class for dataset loaders.
\index{\_\_init\_\_() (graphem.datasets.DatasetLoader method)@\spxentry{\_\_init\_\_()}\spxextra{graphem.datasets.DatasetLoader method}}

\begin{fulllineitems}
\phantomsection\label{\detokenize{graphem:graphem.datasets.DatasetLoader.__init__}}
\pysigstartsignatures
\pysiglinewithargsret{\sphinxbfcode{\sphinxupquote{\_\_init\_\_}}}{\sphinxparam{\DUrole{n,n}{name}}}{}
\pysigstopsignatures
\sphinxAtStartPar
Initialize the dataset loader.
\begin{quote}\begin{description}
\sphinxlineitem{Parameters}
\sphinxAtStartPar
\sphinxstyleliteralstrong{\sphinxupquote{name}} \textendash{} str
Name of the dataset

\end{description}\end{quote}

\end{fulllineitems}

\index{download() (graphem.datasets.DatasetLoader method)@\spxentry{download()}\spxextra{graphem.datasets.DatasetLoader method}}

\begin{fulllineitems}
\phantomsection\label{\detokenize{graphem:graphem.datasets.DatasetLoader.download}}
\pysigstartsignatures
\pysiglinewithargsret{\sphinxbfcode{\sphinxupquote{download}}}{}{}
\pysigstopsignatures
\sphinxAtStartPar
Download the dataset. To be implemented by subclasses.

\end{fulllineitems}

\index{load() (graphem.datasets.DatasetLoader method)@\spxentry{load()}\spxextra{graphem.datasets.DatasetLoader method}}

\begin{fulllineitems}
\phantomsection\label{\detokenize{graphem:graphem.datasets.DatasetLoader.load}}
\pysigstartsignatures
\pysiglinewithargsret{\sphinxbfcode{\sphinxupquote{load}}}{}{}
\pysigstopsignatures
\sphinxAtStartPar
Load the dataset as edges. To be implemented by subclasses.

\end{fulllineitems}

\index{load\_as\_networkx() (graphem.datasets.DatasetLoader method)@\spxentry{load\_as\_networkx()}\spxextra{graphem.datasets.DatasetLoader method}}

\begin{fulllineitems}
\phantomsection\label{\detokenize{graphem:graphem.datasets.DatasetLoader.load_as_networkx}}
\pysigstartsignatures
\pysiglinewithargsret{\sphinxbfcode{\sphinxupquote{load\_as\_networkx}}}{}{}
\pysigstopsignatures
\sphinxAtStartPar
Load the dataset as a NetworkX graph.
\begin{quote}\begin{description}
\sphinxlineitem{Returns}
\sphinxAtStartPar
The loaded graph

\sphinxlineitem{Return type}
\sphinxAtStartPar
\sphinxhref{https://networkx.org/documentation/stable/reference/classes/graph.html\#networkx.Graph}{networkx.Graph}

\end{description}\end{quote}

\end{fulllineitems}

\index{info() (graphem.datasets.DatasetLoader method)@\spxentry{info()}\spxextra{graphem.datasets.DatasetLoader method}}

\begin{fulllineitems}
\phantomsection\label{\detokenize{graphem:graphem.datasets.DatasetLoader.info}}
\pysigstartsignatures
\pysiglinewithargsret{\sphinxbfcode{\sphinxupquote{info}}}{}{}
\pysigstopsignatures
\sphinxAtStartPar
Print information about the dataset.

\end{fulllineitems}

\index{is\_downloaded() (graphem.datasets.DatasetLoader method)@\spxentry{is\_downloaded()}\spxextra{graphem.datasets.DatasetLoader method}}

\begin{fulllineitems}
\phantomsection\label{\detokenize{graphem:graphem.datasets.DatasetLoader.is_downloaded}}
\pysigstartsignatures
\pysiglinewithargsret{\sphinxbfcode{\sphinxupquote{is\_downloaded}}}{}{}
\pysigstopsignatures
\sphinxAtStartPar
Check if the dataset is already downloaded.
\begin{quote}\begin{description}
\sphinxlineitem{Returns}
\sphinxAtStartPar
True if downloaded, False otherwise

\sphinxlineitem{Return type}
\sphinxAtStartPar
\sphinxhref{https://docs.python.org/3/library/functions.html\#bool}{bool}

\end{description}\end{quote}

\end{fulllineitems}


\end{fulllineitems}

\index{SNAPDataset (class in graphem.datasets)@\spxentry{SNAPDataset}\spxextra{class in graphem.datasets}}

\begin{fulllineitems}
\phantomsection\label{\detokenize{graphem:graphem.datasets.SNAPDataset}}
\pysigstartsignatures
\pysiglinewithargsret{\sphinxbfcode{\sphinxupquote{class\DUrole{w,w}{  }}}\sphinxcode{\sphinxupquote{graphem.datasets.}}\sphinxbfcode{\sphinxupquote{SNAPDataset}}}{\sphinxparam{\DUrole{n,n}{dataset\_name}}}{}
\pysigstopsignatures
\sphinxAtStartPar
Bases: {\hyperref[\detokenize{graphem:graphem.datasets.DatasetLoader}]{\sphinxcrossref{\sphinxcode{\sphinxupquote{DatasetLoader}}}}}

\sphinxAtStartPar
Loader for datasets from the Stanford Network Analysis Project (SNAP).

\sphinxAtStartPar
SNAP datasets are commonly used in network analysis research.
Source: \sphinxurl{https://snap.stanford.edu/data/}
\index{AVAILABLE\_DATASETS (graphem.datasets.SNAPDataset attribute)@\spxentry{AVAILABLE\_DATASETS}\spxextra{graphem.datasets.SNAPDataset attribute}}

\begin{fulllineitems}
\phantomsection\label{\detokenize{graphem:graphem.datasets.SNAPDataset.AVAILABLE_DATASETS}}
\pysigstartsignatures
\pysigline{\sphinxbfcode{\sphinxupquote{AVAILABLE\_DATASETS}}\sphinxbfcode{\sphinxupquote{\DUrole{w,w}{  }\DUrole{p,p}{=}\DUrole{w,w}{  }\{\textquotesingle{}ca\sphinxhyphen{}GrQc\textquotesingle{}: \{\textquotesingle{}description\textquotesingle{}: \textquotesingle{}Collaboration network of Arxiv General Relativity\textquotesingle{}, \textquotesingle{}directed\textquotesingle{}: False, \textquotesingle{}edges\textquotesingle{}: 14496, \textquotesingle{}nodes\textquotesingle{}: 5242, \textquotesingle{}url\textquotesingle{}: \textquotesingle{}https://snap.stanford.edu/data/ca\sphinxhyphen{}GrQc.txt.gz\textquotesingle{}\}, \textquotesingle{}ca\sphinxhyphen{}HepTh\textquotesingle{}: \{\textquotesingle{}description\textquotesingle{}: \textquotesingle{}Collaboration network of Arxiv High Energy Physics Theory\textquotesingle{}, \textquotesingle{}directed\textquotesingle{}: False, \textquotesingle{}edges\textquotesingle{}: 25998, \textquotesingle{}nodes\textquotesingle{}: 9877, \textquotesingle{}url\textquotesingle{}: \textquotesingle{}https://snap.stanford.edu/data/ca\sphinxhyphen{}HepTh.txt.gz\textquotesingle{}\}, \textquotesingle{}ego\sphinxhyphen{}twitter\textquotesingle{}: \{\textquotesingle{}description\textquotesingle{}: \textquotesingle{}Twitter ego network\textquotesingle{}, \textquotesingle{}directed\textquotesingle{}: True, \textquotesingle{}edges\textquotesingle{}: 1768149, \textquotesingle{}nodes\textquotesingle{}: 81306, \textquotesingle{}url\textquotesingle{}: \textquotesingle{}https://snap.stanford.edu/data/twitter\_combined.txt.gz\textquotesingle{}\}, \textquotesingle{}email\sphinxhyphen{}Enron\textquotesingle{}: \{\textquotesingle{}description\textquotesingle{}: \textquotesingle{}Email communication network from Enron\textquotesingle{}, \textquotesingle{}directed\textquotesingle{}: True, \textquotesingle{}edges\textquotesingle{}: 183831, \textquotesingle{}nodes\textquotesingle{}: 36692, \textquotesingle{}url\textquotesingle{}: \textquotesingle{}https://snap.stanford.edu/data/email\sphinxhyphen{}Enron.txt.gz\textquotesingle{}\}, \textquotesingle{}facebook\_combined\textquotesingle{}: \{\textquotesingle{}description\textquotesingle{}: \textquotesingle{}Facebook social network\textquotesingle{}, \textquotesingle{}directed\textquotesingle{}: False, \textquotesingle{}edges\textquotesingle{}: 88234, \textquotesingle{}nodes\textquotesingle{}: 4039, \textquotesingle{}url\textquotesingle{}: \textquotesingle{}https://snap.stanford.edu/data/facebook\_combined.txt.gz\textquotesingle{}\}, \textquotesingle{}oregon1\_010331\textquotesingle{}: \{\textquotesingle{}description\textquotesingle{}: \textquotesingle{}AS peering network from Oregon route views\textquotesingle{}, \textquotesingle{}directed\textquotesingle{}: False, \textquotesingle{}edges\textquotesingle{}: 22002, \textquotesingle{}nodes\textquotesingle{}: 10670, \textquotesingle{}url\textquotesingle{}: \textquotesingle{}https://snap.stanford.edu/data/oregon1\_010331.txt.gz\textquotesingle{}\}, \textquotesingle{}p2p\sphinxhyphen{}Gnutella04\textquotesingle{}: \{\textquotesingle{}description\textquotesingle{}: \textquotesingle{}Gnutella peer\sphinxhyphen{}to\sphinxhyphen{}peer network from August 4, 2002\textquotesingle{}, \textquotesingle{}directed\textquotesingle{}: True, \textquotesingle{}edges\textquotesingle{}: 39994, \textquotesingle{}nodes\textquotesingle{}: 10876, \textquotesingle{}url\textquotesingle{}: \textquotesingle{}https://snap.stanford.edu/data/p2p\sphinxhyphen{}Gnutella04.txt.gz\textquotesingle{}\}, \textquotesingle{}wiki\sphinxhyphen{}vote\textquotesingle{}: \{\textquotesingle{}description\textquotesingle{}: \textquotesingle{}Wikipedia who\sphinxhyphen{}votes\sphinxhyphen{}on\sphinxhyphen{}whom network\textquotesingle{}, \textquotesingle{}directed\textquotesingle{}: True, \textquotesingle{}edges\textquotesingle{}: 103689, \textquotesingle{}nodes\textquotesingle{}: 7115, \textquotesingle{}url\textquotesingle{}: \textquotesingle{}https://snap.stanford.edu/data/wiki\sphinxhyphen{}Vote.txt.gz\textquotesingle{}\}\}}}}
\pysigstopsignatures
\end{fulllineitems}

\index{\_\_init\_\_() (graphem.datasets.SNAPDataset method)@\spxentry{\_\_init\_\_()}\spxextra{graphem.datasets.SNAPDataset method}}

\begin{fulllineitems}
\phantomsection\label{\detokenize{graphem:graphem.datasets.SNAPDataset.__init__}}
\pysigstartsignatures
\pysiglinewithargsret{\sphinxbfcode{\sphinxupquote{\_\_init\_\_}}}{\sphinxparam{\DUrole{n,n}{dataset\_name}}}{}
\pysigstopsignatures
\sphinxAtStartPar
Initialize the SNAP dataset loader.
\begin{quote}\begin{description}
\sphinxlineitem{Parameters}
\sphinxAtStartPar
\sphinxstyleliteralstrong{\sphinxupquote{dataset\_name}} \textendash{} str
Name of the SNAP dataset to load

\end{description}\end{quote}

\end{fulllineitems}

\index{download() (graphem.datasets.SNAPDataset method)@\spxentry{download()}\spxextra{graphem.datasets.SNAPDataset method}}

\begin{fulllineitems}
\phantomsection\label{\detokenize{graphem:graphem.datasets.SNAPDataset.download}}
\pysigstartsignatures
\pysiglinewithargsret{\sphinxbfcode{\sphinxupquote{download}}}{}{}
\pysigstopsignatures
\sphinxAtStartPar
Download the SNAP dataset.

\end{fulllineitems}

\index{is\_downloaded() (graphem.datasets.SNAPDataset method)@\spxentry{is\_downloaded()}\spxextra{graphem.datasets.SNAPDataset method}}

\begin{fulllineitems}
\phantomsection\label{\detokenize{graphem:graphem.datasets.SNAPDataset.is_downloaded}}
\pysigstartsignatures
\pysiglinewithargsret{\sphinxbfcode{\sphinxupquote{is\_downloaded}}}{}{}
\pysigstopsignatures
\sphinxAtStartPar
Check if the dataset is already downloaded.

\end{fulllineitems}

\index{load() (graphem.datasets.SNAPDataset method)@\spxentry{load()}\spxextra{graphem.datasets.SNAPDataset method}}

\begin{fulllineitems}
\phantomsection\label{\detokenize{graphem:graphem.datasets.SNAPDataset.load}}
\pysigstartsignatures
\pysiglinewithargsret{\sphinxbfcode{\sphinxupquote{load}}}{}{}
\pysigstopsignatures
\sphinxAtStartPar
Load the SNAP dataset as edges.
\begin{quote}\begin{description}
\sphinxlineitem{Returns}
\sphinxAtStartPar
\begin{description}
\sphinxlineitem{(vertices, edges)}
\sphinxAtStartPar
vertices: np.ndarray of shape (num\_vertices,)
edges: np.ndarray of shape (num\_edges, 2)

\end{description}


\sphinxlineitem{Return type}
\sphinxAtStartPar
\sphinxhref{https://docs.python.org/3/library/stdtypes.html\#tuple}{tuple}

\end{description}\end{quote}

\end{fulllineitems}


\end{fulllineitems}

\index{NetworkRepositoryDataset (class in graphem.datasets)@\spxentry{NetworkRepositoryDataset}\spxextra{class in graphem.datasets}}

\begin{fulllineitems}
\phantomsection\label{\detokenize{graphem:graphem.datasets.NetworkRepositoryDataset}}
\pysigstartsignatures
\pysiglinewithargsret{\sphinxbfcode{\sphinxupquote{class\DUrole{w,w}{  }}}\sphinxcode{\sphinxupquote{graphem.datasets.}}\sphinxbfcode{\sphinxupquote{NetworkRepositoryDataset}}}{\sphinxparam{\DUrole{n,n}{dataset\_name}}}{}
\pysigstopsignatures
\sphinxAtStartPar
Bases: {\hyperref[\detokenize{graphem:graphem.datasets.DatasetLoader}]{\sphinxcrossref{\sphinxcode{\sphinxupquote{DatasetLoader}}}}}

\sphinxAtStartPar
Loader for datasets from the Network Repository.

\sphinxAtStartPar
Network Repository is a scientific network data repository with interactive analytics.
Source: \sphinxurl{https://networkrepository.com/}
\index{AVAILABLE\_DATASETS (graphem.datasets.NetworkRepositoryDataset attribute)@\spxentry{AVAILABLE\_DATASETS}\spxextra{graphem.datasets.NetworkRepositoryDataset attribute}}

\begin{fulllineitems}
\phantomsection\label{\detokenize{graphem:graphem.datasets.NetworkRepositoryDataset.AVAILABLE_DATASETS}}
\pysigstartsignatures
\pysigline{\sphinxbfcode{\sphinxupquote{AVAILABLE\_DATASETS}}\sphinxbfcode{\sphinxupquote{\DUrole{w,w}{  }\DUrole{p,p}{=}\DUrole{w,w}{  }\{\textquotesingle{}ca\sphinxhyphen{}cit\sphinxhyphen{}HepPh\textquotesingle{}: \{\textquotesingle{}description\textquotesingle{}: \textquotesingle{}Citation network of Arxiv High Energy Physics\textquotesingle{}, \textquotesingle{}directed\textquotesingle{}: True, \textquotesingle{}file\_pattern\textquotesingle{}: \textquotesingle{}ca\sphinxhyphen{}cit\sphinxhyphen{}HepPh.mtx\textquotesingle{}, \textquotesingle{}url\textquotesingle{}: \textquotesingle{}https://nrvis.com/download/data/ca/ca\sphinxhyphen{}cit\sphinxhyphen{}HepPh.zip\textquotesingle{}\}, \textquotesingle{}ia\sphinxhyphen{}reality\textquotesingle{}: \{\textquotesingle{}description\textquotesingle{}: \textquotesingle{}Reality Mining social network\textquotesingle{}, \textquotesingle{}directed\textquotesingle{}: False, \textquotesingle{}file\_pattern\textquotesingle{}: \textquotesingle{}ia\sphinxhyphen{}reality.mtx\textquotesingle{}, \textquotesingle{}url\textquotesingle{}: \textquotesingle{}https://nrvis.com/download/data/ia/ia\sphinxhyphen{}reality.zip\textquotesingle{}\}, \textquotesingle{}soc\sphinxhyphen{}hamsterster\textquotesingle{}: \{\textquotesingle{}description\textquotesingle{}: \textquotesingle{}Hamsterster social network\textquotesingle{}, \textquotesingle{}directed\textquotesingle{}: False, \textquotesingle{}file\_pattern\textquotesingle{}: \textquotesingle{}soc\sphinxhyphen{}hamsterster.mtx\textquotesingle{}, \textquotesingle{}url\textquotesingle{}: \textquotesingle{}https://nrvis.com/download/data/soc/soc\sphinxhyphen{}hamsterster.zip\textquotesingle{}\}, \textquotesingle{}socfb\sphinxhyphen{}MIT\textquotesingle{}: \{\textquotesingle{}description\textquotesingle{}: \textquotesingle{}Facebook network from MIT\textquotesingle{}, \textquotesingle{}directed\textquotesingle{}: False, \textquotesingle{}file\_pattern\textquotesingle{}: \textquotesingle{}socfb\sphinxhyphen{}MIT.mtx\textquotesingle{}, \textquotesingle{}url\textquotesingle{}: \textquotesingle{}https://nrvis.com/download/data/socfb/socfb\sphinxhyphen{}MIT.zip\textquotesingle{}\}, \textquotesingle{}web\sphinxhyphen{}google\sphinxhyphen{}dir\textquotesingle{}: \{\textquotesingle{}description\textquotesingle{}: \textquotesingle{}Google web graph\textquotesingle{}, \textquotesingle{}directed\textquotesingle{}: True, \textquotesingle{}file\_pattern\textquotesingle{}: \textquotesingle{}web\sphinxhyphen{}google\sphinxhyphen{}dir.edges\textquotesingle{}, \textquotesingle{}url\textquotesingle{}: \textquotesingle{}https://nrvis.com/download/data/web/web\sphinxhyphen{}google\sphinxhyphen{}dir.zip\textquotesingle{}\}\}}}}
\pysigstopsignatures
\end{fulllineitems}

\index{\_\_init\_\_() (graphem.datasets.NetworkRepositoryDataset method)@\spxentry{\_\_init\_\_()}\spxextra{graphem.datasets.NetworkRepositoryDataset method}}

\begin{fulllineitems}
\phantomsection\label{\detokenize{graphem:graphem.datasets.NetworkRepositoryDataset.__init__}}
\pysigstartsignatures
\pysiglinewithargsret{\sphinxbfcode{\sphinxupquote{\_\_init\_\_}}}{\sphinxparam{\DUrole{n,n}{dataset\_name}}}{}
\pysigstopsignatures
\sphinxAtStartPar
Initialize the Network Repository dataset loader.
\begin{quote}\begin{description}
\sphinxlineitem{Parameters}
\sphinxAtStartPar
\sphinxstyleliteralstrong{\sphinxupquote{dataset\_name}} \textendash{} str
Name of the Network Repository dataset to load

\end{description}\end{quote}

\end{fulllineitems}

\index{download() (graphem.datasets.NetworkRepositoryDataset method)@\spxentry{download()}\spxextra{graphem.datasets.NetworkRepositoryDataset method}}

\begin{fulllineitems}
\phantomsection\label{\detokenize{graphem:graphem.datasets.NetworkRepositoryDataset.download}}
\pysigstartsignatures
\pysiglinewithargsret{\sphinxbfcode{\sphinxupquote{download}}}{}{}
\pysigstopsignatures
\sphinxAtStartPar
Download the Network Repository dataset.

\end{fulllineitems}

\index{is\_downloaded() (graphem.datasets.NetworkRepositoryDataset method)@\spxentry{is\_downloaded()}\spxextra{graphem.datasets.NetworkRepositoryDataset method}}

\begin{fulllineitems}
\phantomsection\label{\detokenize{graphem:graphem.datasets.NetworkRepositoryDataset.is_downloaded}}
\pysigstartsignatures
\pysiglinewithargsret{\sphinxbfcode{\sphinxupquote{is\_downloaded}}}{}{}
\pysigstopsignatures
\sphinxAtStartPar
Check if the expected dataset file exists without scanning the whole tree.
\begin{quote}\begin{description}
\sphinxlineitem{Returns}
\sphinxAtStartPar
True if the expected file exists, False otherwise

\sphinxlineitem{Return type}
\sphinxAtStartPar
\sphinxhref{https://docs.python.org/3/library/functions.html\#bool}{bool}

\end{description}\end{quote}

\end{fulllineitems}

\index{load() (graphem.datasets.NetworkRepositoryDataset method)@\spxentry{load()}\spxextra{graphem.datasets.NetworkRepositoryDataset method}}

\begin{fulllineitems}
\phantomsection\label{\detokenize{graphem:graphem.datasets.NetworkRepositoryDataset.load}}
\pysigstartsignatures
\pysiglinewithargsret{\sphinxbfcode{\sphinxupquote{load}}}{}{}
\pysigstopsignatures
\sphinxAtStartPar
Load the Network Repository dataset as edges.
\begin{quote}\begin{description}
\sphinxlineitem{Returns}
\sphinxAtStartPar
\begin{description}
\sphinxlineitem{(vertices, edges)}
\sphinxAtStartPar
vertices: np.ndarray of shape (num\_vertices,)
edges: np.ndarray of shape (num\_edges, 2)

\end{description}


\sphinxlineitem{Return type}
\sphinxAtStartPar
\sphinxhref{https://docs.python.org/3/library/stdtypes.html\#tuple}{tuple}

\end{description}\end{quote}

\end{fulllineitems}


\end{fulllineitems}

\index{SemanticScholarDataset (class in graphem.datasets)@\spxentry{SemanticScholarDataset}\spxextra{class in graphem.datasets}}

\begin{fulllineitems}
\phantomsection\label{\detokenize{graphem:graphem.datasets.SemanticScholarDataset}}
\pysigstartsignatures
\pysiglinewithargsret{\sphinxbfcode{\sphinxupquote{class\DUrole{w,w}{  }}}\sphinxcode{\sphinxupquote{graphem.datasets.}}\sphinxbfcode{\sphinxupquote{SemanticScholarDataset}}}{\sphinxparam{\DUrole{n,n}{dataset\_name}\DUrole{o,o}{=}\DUrole{default_value}{\textquotesingle{}s2\sphinxhyphen{}CS\textquotesingle{}}}}{}
\pysigstopsignatures
\sphinxAtStartPar
Bases: {\hyperref[\detokenize{graphem:graphem.datasets.DatasetLoader}]{\sphinxcrossref{\sphinxcode{\sphinxupquote{DatasetLoader}}}}}

\sphinxAtStartPar
Loader for Semantic Scholar citation network datasets.

\sphinxAtStartPar
Semantic Scholar is a free, AI\sphinxhyphen{}powered research tool for scientific literature.
This loader downloads and processes the citation network from subsets of Semantic Scholar data.
\index{AVAILABLE\_DATASETS (graphem.datasets.SemanticScholarDataset attribute)@\spxentry{AVAILABLE\_DATASETS}\spxextra{graphem.datasets.SemanticScholarDataset attribute}}

\begin{fulllineitems}
\phantomsection\label{\detokenize{graphem:graphem.datasets.SemanticScholarDataset.AVAILABLE_DATASETS}}
\pysigstartsignatures
\pysigline{\sphinxbfcode{\sphinxupquote{AVAILABLE\_DATASETS}}\sphinxbfcode{\sphinxupquote{\DUrole{w,w}{  }\DUrole{p,p}{=}\DUrole{w,w}{  }\{\textquotesingle{}s2\sphinxhyphen{}CS\textquotesingle{}: \{\textquotesingle{}description\textquotesingle{}: \textquotesingle{}Computer Science citation network from Semantic Scholar\textquotesingle{}, \textquotesingle{}edges\_file\textquotesingle{}: \textquotesingle{}s2\sphinxhyphen{}CS\sphinxhyphen{}citations.csv\textquotesingle{}, \textquotesingle{}nodes\_file\textquotesingle{}: \textquotesingle{}s2\sphinxhyphen{}CS\sphinxhyphen{}nodes.csv\textquotesingle{}, \textquotesingle{}url\textquotesingle{}: \textquotesingle{}https://github.com/mattbierbaum/citation\sphinxhyphen{}networks/raw/master/s2\sphinxhyphen{}CS.tar.gz\textquotesingle{}\}\}}}}
\pysigstopsignatures
\end{fulllineitems}

\index{\_\_init\_\_() (graphem.datasets.SemanticScholarDataset method)@\spxentry{\_\_init\_\_()}\spxextra{graphem.datasets.SemanticScholarDataset method}}

\begin{fulllineitems}
\phantomsection\label{\detokenize{graphem:graphem.datasets.SemanticScholarDataset.__init__}}
\pysigstartsignatures
\pysiglinewithargsret{\sphinxbfcode{\sphinxupquote{\_\_init\_\_}}}{\sphinxparam{\DUrole{n,n}{dataset\_name}\DUrole{o,o}{=}\DUrole{default_value}{\textquotesingle{}s2\sphinxhyphen{}CS\textquotesingle{}}}}{}
\pysigstopsignatures
\sphinxAtStartPar
Initialize the Semantic Scholar dataset loader.
\begin{quote}\begin{description}
\sphinxlineitem{Parameters}
\sphinxAtStartPar
\sphinxstyleliteralstrong{\sphinxupquote{dataset\_name}} \textendash{} str
Name of the Semantic Scholar dataset to load

\end{description}\end{quote}

\end{fulllineitems}

\index{download() (graphem.datasets.SemanticScholarDataset method)@\spxentry{download()}\spxextra{graphem.datasets.SemanticScholarDataset method}}

\begin{fulllineitems}
\phantomsection\label{\detokenize{graphem:graphem.datasets.SemanticScholarDataset.download}}
\pysigstartsignatures
\pysiglinewithargsret{\sphinxbfcode{\sphinxupquote{download}}}{}{}
\pysigstopsignatures
\sphinxAtStartPar
Download the Semantic Scholar dataset.

\end{fulllineitems}

\index{is\_downloaded() (graphem.datasets.SemanticScholarDataset method)@\spxentry{is\_downloaded()}\spxextra{graphem.datasets.SemanticScholarDataset method}}

\begin{fulllineitems}
\phantomsection\label{\detokenize{graphem:graphem.datasets.SemanticScholarDataset.is_downloaded}}
\pysigstartsignatures
\pysiglinewithargsret{\sphinxbfcode{\sphinxupquote{is\_downloaded}}}{}{}
\pysigstopsignatures
\sphinxAtStartPar
Check if the dataset is already downloaded.

\end{fulllineitems}

\index{load() (graphem.datasets.SemanticScholarDataset method)@\spxentry{load()}\spxextra{graphem.datasets.SemanticScholarDataset method}}

\begin{fulllineitems}
\phantomsection\label{\detokenize{graphem:graphem.datasets.SemanticScholarDataset.load}}
\pysigstartsignatures
\pysiglinewithargsret{\sphinxbfcode{\sphinxupquote{load}}}{}{}
\pysigstopsignatures
\sphinxAtStartPar
Load the Semantic Scholar dataset as edges.
\begin{quote}\begin{description}
\sphinxlineitem{Returns}
\sphinxAtStartPar
\begin{description}
\sphinxlineitem{(vertices, edges)}
\sphinxAtStartPar
vertices: np.ndarray of shape (num\_vertices,)
edges: np.ndarray of shape (num\_edges, 2)

\end{description}


\sphinxlineitem{Return type}
\sphinxAtStartPar
\sphinxhref{https://docs.python.org/3/library/stdtypes.html\#tuple}{tuple}

\end{description}\end{quote}

\end{fulllineitems}


\end{fulllineitems}

\index{list\_available\_datasets() (in module graphem.datasets)@\spxentry{list\_available\_datasets()}\spxextra{in module graphem.datasets}}

\begin{fulllineitems}
\phantomsection\label{\detokenize{graphem:graphem.datasets.list_available_datasets}}
\pysigstartsignatures
\pysiglinewithargsret{\sphinxcode{\sphinxupquote{graphem.datasets.}}\sphinxbfcode{\sphinxupquote{list\_available\_datasets}}}{}{}
\pysigstopsignatures
\sphinxAtStartPar
List all available datasets from all sources.
\begin{quote}\begin{description}
\sphinxlineitem{Returns}
\sphinxAtStartPar
Dictionary with dataset information

\sphinxlineitem{Return type}
\sphinxAtStartPar
\sphinxhref{https://docs.python.org/3/library/stdtypes.html\#dict}{dict}

\end{description}\end{quote}

\end{fulllineitems}

\index{load\_dataset() (in module graphem.datasets)@\spxentry{load\_dataset()}\spxextra{in module graphem.datasets}}

\begin{fulllineitems}
\phantomsection\label{\detokenize{graphem:graphem.datasets.load_dataset}}
\pysigstartsignatures
\pysiglinewithargsret{\sphinxcode{\sphinxupquote{graphem.datasets.}}\sphinxbfcode{\sphinxupquote{load\_dataset}}}{\sphinxparam{\DUrole{n,n}{dataset\_name}}}{}
\pysigstopsignatures
\sphinxAtStartPar
Load a dataset by name.
\begin{quote}\begin{description}
\sphinxlineitem{Parameters}
\sphinxAtStartPar
\sphinxstyleliteralstrong{\sphinxupquote{dataset\_name}} \textendash{} str
Name of the dataset to load

\sphinxlineitem{Returns}
\sphinxAtStartPar
\begin{description}
\sphinxlineitem{(vertices, edges)}
\sphinxAtStartPar
vertices: np.ndarray of shape (num\_vertices,)
edges: np.ndarray of shape (num\_edges, 2)

\end{description}


\sphinxlineitem{Return type}
\sphinxAtStartPar
\sphinxhref{https://docs.python.org/3/library/stdtypes.html\#tuple}{tuple}

\end{description}\end{quote}

\end{fulllineitems}

\index{load\_dataset\_as\_networkx() (in module graphem.datasets)@\spxentry{load\_dataset\_as\_networkx()}\spxextra{in module graphem.datasets}}

\begin{fulllineitems}
\phantomsection\label{\detokenize{graphem:graphem.datasets.load_dataset_as_networkx}}
\pysigstartsignatures
\pysiglinewithargsret{\sphinxcode{\sphinxupquote{graphem.datasets.}}\sphinxbfcode{\sphinxupquote{load\_dataset\_as\_networkx}}}{\sphinxparam{\DUrole{n,n}{dataset\_name}}}{}
\pysigstopsignatures
\sphinxAtStartPar
Load a dataset as a NetworkX graph.
\begin{quote}\begin{description}
\sphinxlineitem{Parameters}
\sphinxAtStartPar
\sphinxstyleliteralstrong{\sphinxupquote{dataset\_name}} \textendash{} str
Name of the dataset to load

\sphinxlineitem{Returns}
\sphinxAtStartPar
The loaded graph

\sphinxlineitem{Return type}
\sphinxAtStartPar
\sphinxhref{https://networkx.org/documentation/stable/reference/classes/graph.html\#networkx.Graph}{networkx.Graph}

\end{description}\end{quote}

\end{fulllineitems}



\subsection{graphem.embedder module}
\label{\detokenize{graphem:module-graphem.embedder}}\label{\detokenize{graphem:graphem-embedder-module}}\index{module@\spxentry{module}!graphem.embedder@\spxentry{graphem.embedder}}\index{graphem.embedder@\spxentry{graphem.embedder}!module@\spxentry{module}}
\sphinxAtStartPar
A JAX\sphinxhyphen{}based implementation of graph embedding.
\index{GraphEmbedder (class in graphem.embedder)@\spxentry{GraphEmbedder}\spxextra{class in graphem.embedder}}

\begin{fulllineitems}
\phantomsection\label{\detokenize{graphem:graphem.embedder.GraphEmbedder}}
\pysigstartsignatures
\pysiglinewithargsret{\sphinxbfcode{\sphinxupquote{class\DUrole{w,w}{  }}}\sphinxcode{\sphinxupquote{graphem.embedder.}}\sphinxbfcode{\sphinxupquote{GraphEmbedder}}}{\sphinxparam{\DUrole{n,n}{edges}}\sphinxparamcomma \sphinxparam{\DUrole{n,n}{n\_vertices}}\sphinxparamcomma \sphinxparam{\DUrole{n,n}{dimension}\DUrole{o,o}{=}\DUrole{default_value}{2}}\sphinxparamcomma \sphinxparam{\DUrole{n,n}{L\_min}\DUrole{o,o}{=}\DUrole{default_value}{1.0}}\sphinxparamcomma \sphinxparam{\DUrole{n,n}{k\_attr}\DUrole{o,o}{=}\DUrole{default_value}{0.2}}\sphinxparamcomma \sphinxparam{\DUrole{n,n}{k\_inter}\DUrole{o,o}{=}\DUrole{default_value}{0.5}}\sphinxparamcomma \sphinxparam{\DUrole{n,n}{knn\_k}\DUrole{o,o}{=}\DUrole{default_value}{10}}\sphinxparamcomma \sphinxparam{\DUrole{n,n}{sample\_size}\DUrole{o,o}{=}\DUrole{default_value}{256}}\sphinxparamcomma \sphinxparam{\DUrole{n,n}{batch\_size}\DUrole{o,o}{=}\DUrole{default_value}{1024}}\sphinxparamcomma \sphinxparam{\DUrole{n,n}{my\_logger}\DUrole{o,o}{=}\DUrole{default_value}{None}}\sphinxparamcomma \sphinxparam{\DUrole{n,n}{verbose}\DUrole{o,o}{=}\DUrole{default_value}{True}}}{}
\pysigstopsignatures
\sphinxAtStartPar
Bases: \sphinxhref{https://docs.python.org/3/library/functions.html\#object}{\sphinxcode{\sphinxupquote{object}}}

\sphinxAtStartPar
A class for embedding graphs using the Laplacian embedding.
\index{edges (graphem.embedder.GraphEmbedder attribute)@\spxentry{edges}\spxextra{graphem.embedder.GraphEmbedder attribute}}

\begin{fulllineitems}
\phantomsection\label{\detokenize{graphem:graphem.embedder.GraphEmbedder.edges}}
\pysigstartsignatures
\pysigline{\sphinxbfcode{\sphinxupquote{edges}}}
\pysigstopsignatures
\sphinxAtStartPar
np.ndarray of shape (num\_edges, 2)
Array of edge pairs (i, j) with i \textless{} j.

\end{fulllineitems}

\index{n (graphem.embedder.GraphEmbedder attribute)@\spxentry{n}\spxextra{graphem.embedder.GraphEmbedder attribute}}

\begin{fulllineitems}
\phantomsection\label{\detokenize{graphem:graphem.embedder.GraphEmbedder.n}}
\pysigstartsignatures
\pysigline{\sphinxbfcode{\sphinxupquote{n}}}
\pysigstopsignatures
\sphinxAtStartPar
int
Number of vertices in the graph.

\end{fulllineitems}

\index{dimension (graphem.embedder.GraphEmbedder attribute)@\spxentry{dimension}\spxextra{graphem.embedder.GraphEmbedder attribute}}

\begin{fulllineitems}
\phantomsection\label{\detokenize{graphem:graphem.embedder.GraphEmbedder.dimension}}
\pysigstartsignatures
\pysigline{\sphinxbfcode{\sphinxupquote{dimension}}}
\pysigstopsignatures
\sphinxAtStartPar
int
Dimension of the embedding.

\end{fulllineitems}

\index{L\_min (graphem.embedder.GraphEmbedder attribute)@\spxentry{L\_min}\spxextra{graphem.embedder.GraphEmbedder attribute}}

\begin{fulllineitems}
\phantomsection\label{\detokenize{graphem:graphem.embedder.GraphEmbedder.L_min}}
\pysigstartsignatures
\pysigline{\sphinxbfcode{\sphinxupquote{L\_min}}}
\pysigstopsignatures
\sphinxAtStartPar
float
Minimum length of the spring.

\end{fulllineitems}

\index{k\_attr (graphem.embedder.GraphEmbedder attribute)@\spxentry{k\_attr}\spxextra{graphem.embedder.GraphEmbedder attribute}}

\begin{fulllineitems}
\phantomsection\label{\detokenize{graphem:graphem.embedder.GraphEmbedder.k_attr}}
\pysigstartsignatures
\pysigline{\sphinxbfcode{\sphinxupquote{k\_attr}}}
\pysigstopsignatures
\sphinxAtStartPar
float
Attraction force constant.

\end{fulllineitems}

\index{k\_inter (graphem.embedder.GraphEmbedder attribute)@\spxentry{k\_inter}\spxextra{graphem.embedder.GraphEmbedder attribute}}

\begin{fulllineitems}
\phantomsection\label{\detokenize{graphem:graphem.embedder.GraphEmbedder.k_inter}}
\pysigstartsignatures
\pysigline{\sphinxbfcode{\sphinxupquote{k\_inter}}}
\pysigstopsignatures
\sphinxAtStartPar
float
Repulsion force constant for intersections.

\end{fulllineitems}

\index{knn\_k (graphem.embedder.GraphEmbedder attribute)@\spxentry{knn\_k}\spxextra{graphem.embedder.GraphEmbedder attribute}}

\begin{fulllineitems}
\phantomsection\label{\detokenize{graphem:graphem.embedder.GraphEmbedder.knn_k}}
\pysigstartsignatures
\pysigline{\sphinxbfcode{\sphinxupquote{knn\_k}}}
\pysigstopsignatures
\sphinxAtStartPar
int
Number of nearest neighbors to consider.

\end{fulllineitems}

\index{sample\_size (graphem.embedder.GraphEmbedder attribute)@\spxentry{sample\_size}\spxextra{graphem.embedder.GraphEmbedder attribute}}

\begin{fulllineitems}
\phantomsection\label{\detokenize{graphem:graphem.embedder.GraphEmbedder.sample_size}}
\pysigstartsignatures
\pysigline{\sphinxbfcode{\sphinxupquote{sample\_size}}}
\pysigstopsignatures
\sphinxAtStartPar
int
Number of samples for kNN search.

\end{fulllineitems}

\index{batch\_size (graphem.embedder.GraphEmbedder attribute)@\spxentry{batch\_size}\spxextra{graphem.embedder.GraphEmbedder attribute}}

\begin{fulllineitems}
\phantomsection\label{\detokenize{graphem:graphem.embedder.GraphEmbedder.batch_size}}
\pysigstartsignatures
\pysigline{\sphinxbfcode{\sphinxupquote{batch\_size}}}
\pysigstopsignatures
\sphinxAtStartPar
int
Batch size for kNN search.

\end{fulllineitems}

\index{my\_logger (graphem.embedder.GraphEmbedder attribute)@\spxentry{my\_logger}\spxextra{graphem.embedder.GraphEmbedder attribute}}

\begin{fulllineitems}
\phantomsection\label{\detokenize{graphem:graphem.embedder.GraphEmbedder.my_logger}}
\pysigstartsignatures
\pysigline{\sphinxbfcode{\sphinxupquote{my\_logger}}}
\pysigstopsignatures
\sphinxAtStartPar
loguru.logger
Logger object to use for logging.

\end{fulllineitems}

\index{\_\_init\_\_() (graphem.embedder.GraphEmbedder method)@\spxentry{\_\_init\_\_()}\spxextra{graphem.embedder.GraphEmbedder method}}

\begin{fulllineitems}
\phantomsection\label{\detokenize{graphem:graphem.embedder.GraphEmbedder.__init__}}
\pysigstartsignatures
\pysiglinewithargsret{\sphinxbfcode{\sphinxupquote{\_\_init\_\_}}}{\sphinxparam{\DUrole{n,n}{edges}}\sphinxparamcomma \sphinxparam{\DUrole{n,n}{n\_vertices}}\sphinxparamcomma \sphinxparam{\DUrole{n,n}{dimension}\DUrole{o,o}{=}\DUrole{default_value}{2}}\sphinxparamcomma \sphinxparam{\DUrole{n,n}{L\_min}\DUrole{o,o}{=}\DUrole{default_value}{1.0}}\sphinxparamcomma \sphinxparam{\DUrole{n,n}{k\_attr}\DUrole{o,o}{=}\DUrole{default_value}{0.2}}\sphinxparamcomma \sphinxparam{\DUrole{n,n}{k\_inter}\DUrole{o,o}{=}\DUrole{default_value}{0.5}}\sphinxparamcomma \sphinxparam{\DUrole{n,n}{knn\_k}\DUrole{o,o}{=}\DUrole{default_value}{10}}\sphinxparamcomma \sphinxparam{\DUrole{n,n}{sample\_size}\DUrole{o,o}{=}\DUrole{default_value}{256}}\sphinxparamcomma \sphinxparam{\DUrole{n,n}{batch\_size}\DUrole{o,o}{=}\DUrole{default_value}{1024}}\sphinxparamcomma \sphinxparam{\DUrole{n,n}{my\_logger}\DUrole{o,o}{=}\DUrole{default_value}{None}}\sphinxparamcomma \sphinxparam{\DUrole{n,n}{verbose}\DUrole{o,o}{=}\DUrole{default_value}{True}}}{}
\pysigstopsignatures
\sphinxAtStartPar
Initialize the GraphEmbedder.

\end{fulllineitems}

\index{logger (graphem.embedder.GraphEmbedder attribute)@\spxentry{logger}\spxextra{graphem.embedder.GraphEmbedder attribute}}

\begin{fulllineitems}
\phantomsection\label{\detokenize{graphem:graphem.embedder.GraphEmbedder.logger}}
\pysigstartsignatures
\pysigline{\sphinxbfcode{\sphinxupquote{logger}}}
\pysigstopsignatures
\sphinxAtStartPar
System logger

\end{fulllineitems}

\index{locate\_knn\_midpoints() (graphem.embedder.GraphEmbedder method)@\spxentry{locate\_knn\_midpoints()}\spxextra{graphem.embedder.GraphEmbedder method}}

\begin{fulllineitems}
\phantomsection\label{\detokenize{graphem:graphem.embedder.GraphEmbedder.locate_knn_midpoints}}
\pysigstartsignatures
\pysiglinewithargsret{\sphinxbfcode{\sphinxupquote{locate\_knn\_midpoints}}}{\sphinxparam{\DUrole{n,n}{midpoints}}\sphinxparamcomma \sphinxparam{\DUrole{n,n}{k}}}{}
\pysigstopsignatures
\sphinxAtStartPar
Locate k nearest neighbors for each midpoint.

\end{fulllineitems}

\index{compute\_spring\_forces() (graphem.embedder.GraphEmbedder static method)@\spxentry{compute\_spring\_forces()}\spxextra{graphem.embedder.GraphEmbedder static method}}

\begin{fulllineitems}
\phantomsection\label{\detokenize{graphem:graphem.embedder.GraphEmbedder.compute_spring_forces}}
\pysigstartsignatures
\pysiglinewithargsret{\sphinxbfcode{\sphinxupquote{static\DUrole{w,w}{  }}}\sphinxbfcode{\sphinxupquote{compute\_spring\_forces}}}{\sphinxparam{\DUrole{n,n}{positions}}\sphinxparamcomma \sphinxparam{\DUrole{n,n}{edges}}\sphinxparamcomma \sphinxparam{\DUrole{n,n}{L\_min}}\sphinxparamcomma \sphinxparam{\DUrole{n,n}{k\_attr}}}{}
\pysigstopsignatures
\sphinxAtStartPar
Compute the spring forces between vertices.

\end{fulllineitems}

\index{compute\_intersection\_forces\_with\_knn\_index() (graphem.embedder.GraphEmbedder static method)@\spxentry{compute\_intersection\_forces\_with\_knn\_index()}\spxextra{graphem.embedder.GraphEmbedder static method}}

\begin{fulllineitems}
\phantomsection\label{\detokenize{graphem:graphem.embedder.GraphEmbedder.compute_intersection_forces_with_knn_index}}
\pysigstartsignatures
\pysiglinewithargsret{\sphinxbfcode{\sphinxupquote{static\DUrole{w,w}{  }}}\sphinxbfcode{\sphinxupquote{compute\_intersection\_forces\_with\_knn\_index}}}{\sphinxparam{\DUrole{n,n}{positions}}\sphinxparamcomma \sphinxparam{\DUrole{n,n}{edges}}\sphinxparamcomma \sphinxparam{\DUrole{n,n}{knn\_idx}}\sphinxparamcomma \sphinxparam{\DUrole{n,n}{sampled\_indices}}\sphinxparamcomma \sphinxparam{\DUrole{n,n}{k\_inter}}}{}
\pysigstopsignatures
\sphinxAtStartPar
Compute the intersection forces between vertices and their nearest neighbors.

\end{fulllineitems}

\index{update\_positions() (graphem.embedder.GraphEmbedder method)@\spxentry{update\_positions()}\spxextra{graphem.embedder.GraphEmbedder method}}

\begin{fulllineitems}
\phantomsection\label{\detokenize{graphem:graphem.embedder.GraphEmbedder.update_positions}}
\pysigstartsignatures
\pysiglinewithargsret{\sphinxbfcode{\sphinxupquote{update\_positions}}}{}{}
\pysigstopsignatures
\sphinxAtStartPar
Update the positions of the vertices based on the spring forces and intersection forces.

\end{fulllineitems}

\index{run\_layout() (graphem.embedder.GraphEmbedder method)@\spxentry{run\_layout()}\spxextra{graphem.embedder.GraphEmbedder method}}

\begin{fulllineitems}
\phantomsection\label{\detokenize{graphem:graphem.embedder.GraphEmbedder.run_layout}}
\pysigstartsignatures
\pysiglinewithargsret{\sphinxbfcode{\sphinxupquote{run\_layout}}}{\sphinxparam{\DUrole{n,n}{num\_iterations}\DUrole{o,o}{=}\DUrole{default_value}{100}}}{}
\pysigstopsignatures
\sphinxAtStartPar
Run the layout for a given number of iterations.

\end{fulllineitems}

\index{display\_layout() (graphem.embedder.GraphEmbedder method)@\spxentry{display\_layout()}\spxextra{graphem.embedder.GraphEmbedder method}}

\begin{fulllineitems}
\phantomsection\label{\detokenize{graphem:graphem.embedder.GraphEmbedder.display_layout}}
\pysigstartsignatures
\pysiglinewithargsret{\sphinxbfcode{\sphinxupquote{display\_layout}}}{\sphinxparam{\DUrole{n,n}{edge\_width}\DUrole{o,o}{=}\DUrole{default_value}{1}}\sphinxparamcomma \sphinxparam{\DUrole{n,n}{node\_size}\DUrole{o,o}{=}\DUrole{default_value}{3}}\sphinxparamcomma \sphinxparam{\DUrole{n,n}{node\_colors}\DUrole{o,o}{=}\DUrole{default_value}{None}}}{}
\pysigstopsignatures
\sphinxAtStartPar
Display the graph embedding using Plotly.
\begin{quote}\begin{description}
\sphinxlineitem{Parameters}\begin{itemize}
\item {} 
\sphinxAtStartPar
\sphinxstyleliteralstrong{\sphinxupquote{edge\_width}} (\sphinxhref{https://docs.python.org/3/library/functions.html\#float}{\sphinxstyleliteralemphasis{\sphinxupquote{float}}}) \textendash{} The width of the edges in the graph embedding.

\item {} 
\sphinxAtStartPar
\sphinxstyleliteralstrong{\sphinxupquote{node\_size}} (\sphinxhref{https://docs.python.org/3/library/functions.html\#float}{\sphinxstyleliteralemphasis{\sphinxupquote{float}}}) \textendash{} The size of the nodes in the graph embedding.

\item {} 
\sphinxAtStartPar
\sphinxstyleliteralstrong{\sphinxupquote{node\_colors}} (\sphinxstyleliteralemphasis{\sphinxupquote{array\sphinxhyphen{}like}}\sphinxstyleliteralemphasis{\sphinxupquote{ of }}\sphinxstyleliteralemphasis{\sphinxupquote{shape}}\sphinxstyleliteralemphasis{\sphinxupquote{ (}}\sphinxstyleliteralemphasis{\sphinxupquote{num\_vertices}}\sphinxstyleliteralemphasis{\sphinxupquote{,}}\sphinxstyleliteralemphasis{\sphinxupquote{)}}) \textendash{} An array of colors for each vertex.

\end{itemize}

\sphinxlineitem{Returns}
\sphinxAtStartPar
Displays the graph embedding using Plotly in the appropriate dimension.

\sphinxlineitem{Return type}
\sphinxAtStartPar
None

\end{description}\end{quote}

\end{fulllineitems}


\end{fulllineitems}



\subsection{graphem.generators module}
\label{\detokenize{graphem:module-graphem.generators}}\label{\detokenize{graphem:graphem-generators-module}}\index{module@\spxentry{module}!graphem.generators@\spxentry{graphem.generators}}\index{graphem.generators@\spxentry{graphem.generators}!module@\spxentry{module}}
\sphinxAtStartPar
Graph generators for Graphem.

\sphinxAtStartPar
This module provides various functions to generate different types of graphs.
\index{erdos\_renyi\_graph() (in module graphem.generators)@\spxentry{erdos\_renyi\_graph()}\spxextra{in module graphem.generators}}

\begin{fulllineitems}
\phantomsection\label{\detokenize{graphem:graphem.generators.erdos_renyi_graph}}
\pysigstartsignatures
\pysiglinewithargsret{\sphinxcode{\sphinxupquote{graphem.generators.}}\sphinxbfcode{\sphinxupquote{erdos\_renyi\_graph}}}{\sphinxparam{\DUrole{n,n}{n}}\sphinxparamcomma \sphinxparam{\DUrole{n,n}{p}}\sphinxparamcomma \sphinxparam{\DUrole{n,n}{seed}\DUrole{o,o}{=}\DUrole{default_value}{0}}}{}
\pysigstopsignatures
\sphinxAtStartPar
Generate a random undirected graph using the Erdős\textendash{}Rényi G(n, p) model.
\begin{quote}\begin{description}
\sphinxlineitem{Parameters}\begin{itemize}
\item {} 
\sphinxAtStartPar
\sphinxstyleliteralstrong{\sphinxupquote{n}} \textendash{} int
Number of vertices.

\item {} 
\sphinxAtStartPar
\sphinxstyleliteralstrong{\sphinxupquote{p}} \textendash{} float
Probability that an edge exists between any pair of vertices.

\item {} 
\sphinxAtStartPar
\sphinxstyleliteralstrong{\sphinxupquote{seed}} \textendash{} int
Random seed for reproducibility.

\end{itemize}

\sphinxlineitem{Returns}
\sphinxAtStartPar
\begin{description}
\sphinxlineitem{np.ndarray of shape (num\_edges, 2)}
\sphinxAtStartPar
Array of edge pairs (i, j) with i \textless{} j.

\end{description}


\sphinxlineitem{Return type}
\sphinxAtStartPar
edges

\end{description}\end{quote}

\end{fulllineitems}

\index{compute\_vertex\_degrees() (in module graphem.generators)@\spxentry{compute\_vertex\_degrees()}\spxextra{in module graphem.generators}}

\begin{fulllineitems}
\phantomsection\label{\detokenize{graphem:graphem.generators.compute_vertex_degrees}}
\pysigstartsignatures
\pysiglinewithargsret{\sphinxcode{\sphinxupquote{graphem.generators.}}\sphinxbfcode{\sphinxupquote{compute\_vertex\_degrees}}}{\sphinxparam{\DUrole{n,n}{n}}\sphinxparamcomma \sphinxparam{\DUrole{n,n}{edges}}}{}
\pysigstopsignatures
\sphinxAtStartPar
Compute the degree of each vertex from the edge list.
\begin{quote}\begin{description}
\sphinxlineitem{Parameters}\begin{itemize}
\item {} 
\sphinxAtStartPar
\sphinxstyleliteralstrong{\sphinxupquote{n}} \textendash{} number of vertices

\item {} 
\sphinxAtStartPar
\sphinxstyleliteralstrong{\sphinxupquote{edges}} \textendash{} array of shape (num\_edges, 2)

\end{itemize}

\sphinxlineitem{Returns}
\sphinxAtStartPar
np.array of shape (n,) with degree of each vertex

\sphinxlineitem{Return type}
\sphinxAtStartPar
degrees

\end{description}\end{quote}

\end{fulllineitems}

\index{generate\_sbm() (in module graphem.generators)@\spxentry{generate\_sbm()}\spxextra{in module graphem.generators}}

\begin{fulllineitems}
\phantomsection\label{\detokenize{graphem:graphem.generators.generate_sbm}}
\pysigstartsignatures
\pysiglinewithargsret{\sphinxcode{\sphinxupquote{graphem.generators.}}\sphinxbfcode{\sphinxupquote{generate\_sbm}}}{\sphinxparam{\DUrole{n,n}{n\_per\_block}\DUrole{o,o}{=}\DUrole{default_value}{75}}\sphinxparamcomma \sphinxparam{\DUrole{n,n}{num\_blocks}\DUrole{o,o}{=}\DUrole{default_value}{4}}\sphinxparamcomma \sphinxparam{\DUrole{n,n}{p\_in}\DUrole{o,o}{=}\DUrole{default_value}{0.15}}\sphinxparamcomma \sphinxparam{\DUrole{n,n}{p\_out}\DUrole{o,o}{=}\DUrole{default_value}{0.01}}\sphinxparamcomma \sphinxparam{\DUrole{n,n}{labels}\DUrole{o,o}{=}\DUrole{default_value}{False}}\sphinxparamcomma \sphinxparam{\DUrole{n,n}{seed}\DUrole{o,o}{=}\DUrole{default_value}{0}}}{}
\pysigstopsignatures
\sphinxAtStartPar
Generate a stochastic block model graph.
\begin{quote}\begin{description}
\sphinxlineitem{Parameters}\begin{itemize}
\item {} 
\sphinxAtStartPar
\sphinxstyleliteralstrong{\sphinxupquote{n\_per\_block}} \textendash{} int
Number of vertices per block.

\item {} 
\sphinxAtStartPar
\sphinxstyleliteralstrong{\sphinxupquote{num\_blocks}} \textendash{} int
Number of blocks.

\item {} 
\sphinxAtStartPar
\sphinxstyleliteralstrong{\sphinxupquote{p\_in}} \textendash{} float
Probability of edge within a block.

\item {} 
\sphinxAtStartPar
\sphinxstyleliteralstrong{\sphinxupquote{p\_out}} \textendash{} float
Probability of edge between blocks.

\item {} 
\sphinxAtStartPar
\sphinxstyleliteralstrong{\sphinxupquote{labels}} \textendash{} bool
If True, return vertex labels.

\item {} 
\sphinxAtStartPar
\sphinxstyleliteralstrong{\sphinxupquote{seed}} \textendash{} int
Random seed for reproducibility.

\end{itemize}

\sphinxlineitem{Returns}
\sphinxAtStartPar
\begin{description}
\sphinxlineitem{np.ndarray of shape (num\_edges, 2)}
\sphinxAtStartPar
Array of edge pairs (i, j) with i \textless{} j.

\sphinxlineitem{labels: np.ndarray of shape (n,) (only if labels=True)}
\sphinxAtStartPar
Block labels for each vertex.

\end{description}


\sphinxlineitem{Return type}
\sphinxAtStartPar
edges

\end{description}\end{quote}

\end{fulllineitems}

\index{generate\_ba() (in module graphem.generators)@\spxentry{generate\_ba()}\spxextra{in module graphem.generators}}

\begin{fulllineitems}
\phantomsection\label{\detokenize{graphem:graphem.generators.generate_ba}}
\pysigstartsignatures
\pysiglinewithargsret{\sphinxcode{\sphinxupquote{graphem.generators.}}\sphinxbfcode{\sphinxupquote{generate\_ba}}}{\sphinxparam{\DUrole{n,n}{n}\DUrole{o,o}{=}\DUrole{default_value}{300}}\sphinxparamcomma \sphinxparam{\DUrole{n,n}{m}\DUrole{o,o}{=}\DUrole{default_value}{3}}\sphinxparamcomma \sphinxparam{\DUrole{n,n}{seed}\DUrole{o,o}{=}\DUrole{default_value}{0}}}{}
\pysigstopsignatures
\sphinxAtStartPar
Generate a Barabási\sphinxhyphen{}Albert preferential attachment graph.
\begin{quote}\begin{description}
\sphinxlineitem{Parameters}\begin{itemize}
\item {} 
\sphinxAtStartPar
\sphinxstyleliteralstrong{\sphinxupquote{n}} \textendash{} int
Number of vertices.

\item {} 
\sphinxAtStartPar
\sphinxstyleliteralstrong{\sphinxupquote{m}} \textendash{} int
Number of edges to attach from a new vertex to existing vertices.

\item {} 
\sphinxAtStartPar
\sphinxstyleliteralstrong{\sphinxupquote{seed}} \textendash{} int
Random seed for reproducibility.

\end{itemize}

\sphinxlineitem{Returns}
\sphinxAtStartPar
\begin{description}
\sphinxlineitem{np.ndarray of shape (num\_edges, 2)}
\sphinxAtStartPar
Array of edge pairs (i, j) with i \textless{} j.

\end{description}


\sphinxlineitem{Return type}
\sphinxAtStartPar
edges

\end{description}\end{quote}

\end{fulllineitems}

\index{generate\_ws() (in module graphem.generators)@\spxentry{generate\_ws()}\spxextra{in module graphem.generators}}

\begin{fulllineitems}
\phantomsection\label{\detokenize{graphem:graphem.generators.generate_ws}}
\pysigstartsignatures
\pysiglinewithargsret{\sphinxcode{\sphinxupquote{graphem.generators.}}\sphinxbfcode{\sphinxupquote{generate\_ws}}}{\sphinxparam{\DUrole{n,n}{n}\DUrole{o,o}{=}\DUrole{default_value}{1000}}\sphinxparamcomma \sphinxparam{\DUrole{n,n}{k}\DUrole{o,o}{=}\DUrole{default_value}{6}}\sphinxparamcomma \sphinxparam{\DUrole{n,n}{p}\DUrole{o,o}{=}\DUrole{default_value}{0.3}}\sphinxparamcomma \sphinxparam{\DUrole{n,n}{seed}\DUrole{o,o}{=}\DUrole{default_value}{0}}}{}
\pysigstopsignatures
\sphinxAtStartPar
Generate a Watts\sphinxhyphen{}Strogatz small\sphinxhyphen{}world graph.
\begin{quote}\begin{description}
\sphinxlineitem{Parameters}\begin{itemize}
\item {} 
\sphinxAtStartPar
\sphinxstyleliteralstrong{\sphinxupquote{n}} \textendash{} int
Number of vertices.

\item {} 
\sphinxAtStartPar
\sphinxstyleliteralstrong{\sphinxupquote{k}} \textendash{} int
Each vertex is connected to k nearest neighbors in ring topology.

\item {} 
\sphinxAtStartPar
\sphinxstyleliteralstrong{\sphinxupquote{p}} \textendash{} float
Probability of rewiring each edge.

\item {} 
\sphinxAtStartPar
\sphinxstyleliteralstrong{\sphinxupquote{seed}} \textendash{} int
Random seed for reproducibility.

\end{itemize}

\sphinxlineitem{Returns}
\sphinxAtStartPar
\begin{description}
\sphinxlineitem{np.ndarray of shape (num\_edges, 2)}
\sphinxAtStartPar
Array of edge pairs (i, j) with i \textless{} j.

\end{description}


\sphinxlineitem{Return type}
\sphinxAtStartPar
edges

\end{description}\end{quote}

\end{fulllineitems}

\index{generate\_power\_cluster() (in module graphem.generators)@\spxentry{generate\_power\_cluster()}\spxextra{in module graphem.generators}}

\begin{fulllineitems}
\phantomsection\label{\detokenize{graphem:graphem.generators.generate_power_cluster}}
\pysigstartsignatures
\pysiglinewithargsret{\sphinxcode{\sphinxupquote{graphem.generators.}}\sphinxbfcode{\sphinxupquote{generate\_power\_cluster}}}{\sphinxparam{\DUrole{n,n}{n}\DUrole{o,o}{=}\DUrole{default_value}{1000}}\sphinxparamcomma \sphinxparam{\DUrole{n,n}{m}\DUrole{o,o}{=}\DUrole{default_value}{3}}\sphinxparamcomma \sphinxparam{\DUrole{n,n}{p}\DUrole{o,o}{=}\DUrole{default_value}{0.5}}\sphinxparamcomma \sphinxparam{\DUrole{n,n}{seed}\DUrole{o,o}{=}\DUrole{default_value}{0}}}{}
\pysigstopsignatures
\sphinxAtStartPar
Generate a powerlaw cluster graph.
\begin{quote}\begin{description}
\sphinxlineitem{Parameters}\begin{itemize}
\item {} 
\sphinxAtStartPar
\sphinxstyleliteralstrong{\sphinxupquote{n}} \textendash{} int
Number of vertices.

\item {} 
\sphinxAtStartPar
\sphinxstyleliteralstrong{\sphinxupquote{m}} \textendash{} int
Number of random edges to add per new vertex.

\item {} 
\sphinxAtStartPar
\sphinxstyleliteralstrong{\sphinxupquote{p}} \textendash{} float
Probability of adding a triangle after adding a random edge.

\item {} 
\sphinxAtStartPar
\sphinxstyleliteralstrong{\sphinxupquote{seed}} \textendash{} int
Random seed for reproducibility.

\end{itemize}

\sphinxlineitem{Returns}
\sphinxAtStartPar
\begin{description}
\sphinxlineitem{np.ndarray of shape (num\_edges, 2)}
\sphinxAtStartPar
Array of edge pairs (i, j) with i \textless{} j.

\end{description}


\sphinxlineitem{Return type}
\sphinxAtStartPar
edges

\end{description}\end{quote}

\end{fulllineitems}

\index{generate\_road\_network() (in module graphem.generators)@\spxentry{generate\_road\_network()}\spxextra{in module graphem.generators}}

\begin{fulllineitems}
\phantomsection\label{\detokenize{graphem:graphem.generators.generate_road_network}}
\pysigstartsignatures
\pysiglinewithargsret{\sphinxcode{\sphinxupquote{graphem.generators.}}\sphinxbfcode{\sphinxupquote{generate\_road\_network}}}{\sphinxparam{\DUrole{n,n}{width}\DUrole{o,o}{=}\DUrole{default_value}{30}}\sphinxparamcomma \sphinxparam{\DUrole{n,n}{height}\DUrole{o,o}{=}\DUrole{default_value}{30}}}{}
\pysigstopsignatures
\sphinxAtStartPar
Generate a 2D grid graph representing a road network.
\begin{quote}\begin{description}
\sphinxlineitem{Parameters}\begin{itemize}
\item {} 
\sphinxAtStartPar
\sphinxstyleliteralstrong{\sphinxupquote{width}} \textendash{} int
Width of the grid.

\item {} 
\sphinxAtStartPar
\sphinxstyleliteralstrong{\sphinxupquote{height}} \textendash{} int
Height of the grid.

\end{itemize}

\sphinxlineitem{Returns}
\sphinxAtStartPar
\begin{description}
\sphinxlineitem{np.ndarray of shape (num\_edges, 2)}
\sphinxAtStartPar
Array of edge pairs (i, j) with i \textless{} j.

\end{description}


\sphinxlineitem{Return type}
\sphinxAtStartPar
edges

\end{description}\end{quote}

\end{fulllineitems}

\index{generate\_bipartite\_graph() (in module graphem.generators)@\spxentry{generate\_bipartite\_graph()}\spxextra{in module graphem.generators}}

\begin{fulllineitems}
\phantomsection\label{\detokenize{graphem:graphem.generators.generate_bipartite_graph}}
\pysigstartsignatures
\pysiglinewithargsret{\sphinxcode{\sphinxupquote{graphem.generators.}}\sphinxbfcode{\sphinxupquote{generate\_bipartite\_graph}}}{\sphinxparam{\DUrole{n,n}{n\_top}\DUrole{o,o}{=}\DUrole{default_value}{50}}\sphinxparamcomma \sphinxparam{\DUrole{n,n}{n\_bottom}\DUrole{o,o}{=}\DUrole{default_value}{100}}}{}
\pysigstopsignatures
\sphinxAtStartPar
Generate a random bipartite graph.
\begin{quote}\begin{description}
\sphinxlineitem{Parameters}\begin{itemize}
\item {} 
\sphinxAtStartPar
\sphinxstyleliteralstrong{\sphinxupquote{n\_top}} \textendash{} int
Number of vertices in the top set.

\item {} 
\sphinxAtStartPar
\sphinxstyleliteralstrong{\sphinxupquote{n\_bottom}} \textendash{} int
Number of vertices in the bottom set.

\end{itemize}

\sphinxlineitem{Returns}
\sphinxAtStartPar
\begin{description}
\sphinxlineitem{np.ndarray of shape (num\_edges, 2)}
\sphinxAtStartPar
Array of edge pairs (i, j) with i \textless{} j.

\end{description}


\sphinxlineitem{Return type}
\sphinxAtStartPar
edges

\end{description}\end{quote}

\end{fulllineitems}

\index{generate\_balanced\_tree() (in module graphem.generators)@\spxentry{generate\_balanced\_tree()}\spxextra{in module graphem.generators}}

\begin{fulllineitems}
\phantomsection\label{\detokenize{graphem:graphem.generators.generate_balanced_tree}}
\pysigstartsignatures
\pysiglinewithargsret{\sphinxcode{\sphinxupquote{graphem.generators.}}\sphinxbfcode{\sphinxupquote{generate\_balanced\_tree}}}{\sphinxparam{\DUrole{n,n}{r}\DUrole{o,o}{=}\DUrole{default_value}{2}}\sphinxparamcomma \sphinxparam{\DUrole{n,n}{h}\DUrole{o,o}{=}\DUrole{default_value}{10}}}{}
\pysigstopsignatures
\sphinxAtStartPar
Generate a balanced r\sphinxhyphen{}ary tree of height h.
\begin{quote}\begin{description}
\sphinxlineitem{Parameters}\begin{itemize}
\item {} 
\sphinxAtStartPar
\sphinxstyleliteralstrong{\sphinxupquote{r}} \textendash{} int
Branching factor of the tree.

\item {} 
\sphinxAtStartPar
\sphinxstyleliteralstrong{\sphinxupquote{h}} \textendash{} int
Height of the tree.

\end{itemize}

\sphinxlineitem{Returns}
\sphinxAtStartPar
\begin{description}
\sphinxlineitem{np.ndarray of shape (num\_edges, 2)}
\sphinxAtStartPar
Array of edge pairs (i, j) with i \textless{} j.

\end{description}


\sphinxlineitem{Return type}
\sphinxAtStartPar
edges

\end{description}\end{quote}

\end{fulllineitems}

\index{generate\_random\_regular() (in module graphem.generators)@\spxentry{generate\_random\_regular()}\spxextra{in module graphem.generators}}

\begin{fulllineitems}
\phantomsection\label{\detokenize{graphem:graphem.generators.generate_random_regular}}
\pysigstartsignatures
\pysiglinewithargsret{\sphinxcode{\sphinxupquote{graphem.generators.}}\sphinxbfcode{\sphinxupquote{generate\_random\_regular}}}{\sphinxparam{\DUrole{n,n}{n}\DUrole{o,o}{=}\DUrole{default_value}{100}}\sphinxparamcomma \sphinxparam{\DUrole{n,n}{d}\DUrole{o,o}{=}\DUrole{default_value}{3}}\sphinxparamcomma \sphinxparam{\DUrole{n,n}{seed}\DUrole{o,o}{=}\DUrole{default_value}{0}}}{}
\pysigstopsignatures
\sphinxAtStartPar
Generate a random regular graph where each node has degree d.
\begin{quote}\begin{description}
\sphinxlineitem{Parameters}\begin{itemize}
\item {} 
\sphinxAtStartPar
\sphinxstyleliteralstrong{\sphinxupquote{n}} \textendash{} int
Number of vertices.

\item {} 
\sphinxAtStartPar
\sphinxstyleliteralstrong{\sphinxupquote{d}} \textendash{} int
Degree of each vertex.

\item {} 
\sphinxAtStartPar
\sphinxstyleliteralstrong{\sphinxupquote{seed}} \textendash{} int
Random seed for reproducibility.

\end{itemize}

\sphinxlineitem{Returns}
\sphinxAtStartPar
\begin{description}
\sphinxlineitem{np.ndarray of shape (num\_edges, 2)}
\sphinxAtStartPar
Array of edge pairs (i, j) with i \textless{} j.

\end{description}


\sphinxlineitem{Return type}
\sphinxAtStartPar
edges

\end{description}\end{quote}

\end{fulllineitems}

\index{generate\_scale\_free() (in module graphem.generators)@\spxentry{generate\_scale\_free()}\spxextra{in module graphem.generators}}

\begin{fulllineitems}
\phantomsection\label{\detokenize{graphem:graphem.generators.generate_scale_free}}
\pysigstartsignatures
\pysiglinewithargsret{\sphinxcode{\sphinxupquote{graphem.generators.}}\sphinxbfcode{\sphinxupquote{generate\_scale\_free}}}{\sphinxparam{\DUrole{n,n}{n}\DUrole{o,o}{=}\DUrole{default_value}{100}}\sphinxparamcomma \sphinxparam{\DUrole{n,n}{alpha}\DUrole{o,o}{=}\DUrole{default_value}{0.41}}\sphinxparamcomma \sphinxparam{\DUrole{n,n}{beta}\DUrole{o,o}{=}\DUrole{default_value}{0.54}}\sphinxparamcomma \sphinxparam{\DUrole{n,n}{gamma}\DUrole{o,o}{=}\DUrole{default_value}{0.05}}\sphinxparamcomma \sphinxparam{\DUrole{n,n}{delta\_in}\DUrole{o,o}{=}\DUrole{default_value}{0.2}}\sphinxparamcomma \sphinxparam{\DUrole{n,n}{delta\_out}\DUrole{o,o}{=}\DUrole{default_value}{0}}\sphinxparamcomma \sphinxparam{\DUrole{n,n}{seed}\DUrole{o,o}{=}\DUrole{default_value}{0}}}{}
\pysigstopsignatures
\sphinxAtStartPar
Generate a scale\sphinxhyphen{}free graph using Holme and Kim algorithm.
\begin{quote}\begin{description}
\sphinxlineitem{Parameters}\begin{itemize}
\item {} 
\sphinxAtStartPar
\sphinxstyleliteralstrong{\sphinxupquote{n}} \textendash{} int
Number of vertices.

\item {} 
\sphinxAtStartPar
\sphinxstyleliteralstrong{\sphinxupquote{alpha}} \textendash{} float
Parameters for the scale\sphinxhyphen{}free graph generation.

\item {} 
\sphinxAtStartPar
\sphinxstyleliteralstrong{\sphinxupquote{beta}} \textendash{} float
Parameters for the scale\sphinxhyphen{}free graph generation.

\item {} 
\sphinxAtStartPar
\sphinxstyleliteralstrong{\sphinxupquote{gamma}} \textendash{} float
Parameters for the scale\sphinxhyphen{}free graph generation.

\item {} 
\sphinxAtStartPar
\sphinxstyleliteralstrong{\sphinxupquote{delta\_in}} \textendash{} float
Parameters for the scale\sphinxhyphen{}free graph generation.

\item {} 
\sphinxAtStartPar
\sphinxstyleliteralstrong{\sphinxupquote{delta\_out}} \textendash{} float
Parameters for the scale\sphinxhyphen{}free graph generation.

\item {} 
\sphinxAtStartPar
\sphinxstyleliteralstrong{\sphinxupquote{seed}} \textendash{} int
Random seed for reproducibility.

\end{itemize}

\sphinxlineitem{Returns}
\sphinxAtStartPar
\begin{description}
\sphinxlineitem{np.ndarray of shape (num\_edges, 2)}
\sphinxAtStartPar
Array of edge pairs (i, j) with i \textless{} j.

\end{description}


\sphinxlineitem{Return type}
\sphinxAtStartPar
edges

\end{description}\end{quote}

\end{fulllineitems}

\index{generate\_geometric() (in module graphem.generators)@\spxentry{generate\_geometric()}\spxextra{in module graphem.generators}}

\begin{fulllineitems}
\phantomsection\label{\detokenize{graphem:graphem.generators.generate_geometric}}
\pysigstartsignatures
\pysiglinewithargsret{\sphinxcode{\sphinxupquote{graphem.generators.}}\sphinxbfcode{\sphinxupquote{generate\_geometric}}}{\sphinxparam{\DUrole{n,n}{n}\DUrole{o,o}{=}\DUrole{default_value}{100}}\sphinxparamcomma \sphinxparam{\DUrole{n,n}{radius}\DUrole{o,o}{=}\DUrole{default_value}{0.2}}\sphinxparamcomma \sphinxparam{\DUrole{n,n}{dim}\DUrole{o,o}{=}\DUrole{default_value}{2}}\sphinxparamcomma \sphinxparam{\DUrole{n,n}{seed}\DUrole{o,o}{=}\DUrole{default_value}{0}}}{}
\pysigstopsignatures
\sphinxAtStartPar
Generate a random geometric graph in a unit cube.
\begin{quote}\begin{description}
\sphinxlineitem{Parameters}\begin{itemize}
\item {} 
\sphinxAtStartPar
\sphinxstyleliteralstrong{\sphinxupquote{n}} \textendash{} int
Number of vertices.

\item {} 
\sphinxAtStartPar
\sphinxstyleliteralstrong{\sphinxupquote{radius}} \textendash{} float
Distance threshold for connecting vertices.

\item {} 
\sphinxAtStartPar
\sphinxstyleliteralstrong{\sphinxupquote{dim}} \textendash{} int
Dimension of the space.

\item {} 
\sphinxAtStartPar
\sphinxstyleliteralstrong{\sphinxupquote{seed}} \textendash{} int
Random seed for reproducibility.

\end{itemize}

\sphinxlineitem{Returns}
\sphinxAtStartPar
\begin{description}
\sphinxlineitem{np.ndarray of shape (num\_edges, 2)}
\sphinxAtStartPar
Array of edge pairs (i, j) with i \textless{} j.

\end{description}


\sphinxlineitem{Return type}
\sphinxAtStartPar
edges

\end{description}\end{quote}

\end{fulllineitems}

\index{generate\_caveman() (in module graphem.generators)@\spxentry{generate\_caveman()}\spxextra{in module graphem.generators}}

\begin{fulllineitems}
\phantomsection\label{\detokenize{graphem:graphem.generators.generate_caveman}}
\pysigstartsignatures
\pysiglinewithargsret{\sphinxcode{\sphinxupquote{graphem.generators.}}\sphinxbfcode{\sphinxupquote{generate\_caveman}}}{\sphinxparam{\DUrole{n,n}{l}\DUrole{o,o}{=}\DUrole{default_value}{10}}\sphinxparamcomma \sphinxparam{\DUrole{n,n}{k}\DUrole{o,o}{=}\DUrole{default_value}{10}}}{}
\pysigstopsignatures
\sphinxAtStartPar
Generate a caveman graph with l cliques of size k.
\begin{quote}\begin{description}
\sphinxlineitem{Parameters}\begin{itemize}
\item {} 
\sphinxAtStartPar
\sphinxstyleliteralstrong{\sphinxupquote{l}} \textendash{} int
Number of cliques.

\item {} 
\sphinxAtStartPar
\sphinxstyleliteralstrong{\sphinxupquote{k}} \textendash{} int
Size of each clique.

\end{itemize}

\sphinxlineitem{Returns}
\sphinxAtStartPar
\begin{description}
\sphinxlineitem{np.ndarray of shape (num\_edges, 2)}
\sphinxAtStartPar
Array of edge pairs (i, j) with i \textless{} j.

\end{description}


\sphinxlineitem{Return type}
\sphinxAtStartPar
edges

\end{description}\end{quote}

\end{fulllineitems}

\index{generate\_relaxed\_caveman() (in module graphem.generators)@\spxentry{generate\_relaxed\_caveman()}\spxextra{in module graphem.generators}}

\begin{fulllineitems}
\phantomsection\label{\detokenize{graphem:graphem.generators.generate_relaxed_caveman}}
\pysigstartsignatures
\pysiglinewithargsret{\sphinxcode{\sphinxupquote{graphem.generators.}}\sphinxbfcode{\sphinxupquote{generate\_relaxed\_caveman}}}{\sphinxparam{\DUrole{n,n}{l}\DUrole{o,o}{=}\DUrole{default_value}{10}}\sphinxparamcomma \sphinxparam{\DUrole{n,n}{k}\DUrole{o,o}{=}\DUrole{default_value}{10}}\sphinxparamcomma \sphinxparam{\DUrole{n,n}{p}\DUrole{o,o}{=}\DUrole{default_value}{0.1}}\sphinxparamcomma \sphinxparam{\DUrole{n,n}{seed}\DUrole{o,o}{=}\DUrole{default_value}{0}}}{}
\pysigstopsignatures
\sphinxAtStartPar
Generate a relaxed caveman graph with l cliques of size k,
and a rewiring probability p.
\begin{quote}\begin{description}
\sphinxlineitem{Parameters}\begin{itemize}
\item {} 
\sphinxAtStartPar
\sphinxstyleliteralstrong{\sphinxupquote{l}} \textendash{} int
Number of cliques.

\item {} 
\sphinxAtStartPar
\sphinxstyleliteralstrong{\sphinxupquote{k}} \textendash{} int
Size of each clique.

\item {} 
\sphinxAtStartPar
\sphinxstyleliteralstrong{\sphinxupquote{p}} \textendash{} float
Rewiring probability.

\item {} 
\sphinxAtStartPar
\sphinxstyleliteralstrong{\sphinxupquote{seed}} \textendash{} int
Random seed for reproducibility.

\end{itemize}

\sphinxlineitem{Returns}
\sphinxAtStartPar
\begin{description}
\sphinxlineitem{np.ndarray of shape (num\_edges, 2)}
\sphinxAtStartPar
Array of edge pairs (i, j) with i \textless{} j.

\end{description}


\sphinxlineitem{Return type}
\sphinxAtStartPar
edges

\end{description}\end{quote}

\end{fulllineitems}



\subsection{graphem.index module}
\label{\detokenize{graphem:module-graphem.index}}\label{\detokenize{graphem:graphem-index-module}}\index{module@\spxentry{module}!graphem.index@\spxentry{graphem.index}}\index{graphem.index@\spxentry{graphem.index}!module@\spxentry{module}}
\sphinxAtStartPar
A JAX\sphinxhyphen{}based implementation of efficient k\sphinxhyphen{}nearest neighbors.
\index{HPIndex (class in graphem.index)@\spxentry{HPIndex}\spxextra{class in graphem.index}}

\begin{fulllineitems}
\phantomsection\label{\detokenize{graphem:graphem.index.HPIndex}}
\pysigstartsignatures
\pysigline{\sphinxbfcode{\sphinxupquote{class\DUrole{w,w}{  }}}\sphinxcode{\sphinxupquote{graphem.index.}}\sphinxbfcode{\sphinxupquote{HPIndex}}}
\pysigstopsignatures
\sphinxAtStartPar
Bases: \sphinxhref{https://docs.python.org/3/library/functions.html\#object}{\sphinxcode{\sphinxupquote{object}}}

\sphinxAtStartPar
A kernelized kNN index that uses batching / tiling to efficiently handle
large datasets with limited memory usage.
\index{\_\_init\_\_() (graphem.index.HPIndex method)@\spxentry{\_\_init\_\_()}\spxextra{graphem.index.HPIndex method}}

\begin{fulllineitems}
\phantomsection\label{\detokenize{graphem:graphem.index.HPIndex.__init__}}
\pysigstartsignatures
\pysiglinewithargsret{\sphinxbfcode{\sphinxupquote{\_\_init\_\_}}}{}{}
\pysigstopsignatures
\end{fulllineitems}

\index{knn\_tiled() (graphem.index.HPIndex static method)@\spxentry{knn\_tiled()}\spxextra{graphem.index.HPIndex static method}}

\begin{fulllineitems}
\phantomsection\label{\detokenize{graphem:graphem.index.HPIndex.knn_tiled}}
\pysigstartsignatures
\pysiglinewithargsret{\sphinxbfcode{\sphinxupquote{static\DUrole{w,w}{  }}}\sphinxbfcode{\sphinxupquote{knn\_tiled}}}{\sphinxparam{\DUrole{n,n}{x}}\sphinxparamcomma \sphinxparam{\DUrole{n,n}{y}}\sphinxparamcomma \sphinxparam{\DUrole{n,n}{k}\DUrole{o,o}{=}\DUrole{default_value}{5}}\sphinxparamcomma \sphinxparam{\DUrole{n,n}{x\_tile\_size}\DUrole{o,o}{=}\DUrole{default_value}{8192}}\sphinxparamcomma \sphinxparam{\DUrole{n,n}{y\_batch\_size}\DUrole{o,o}{=}\DUrole{default_value}{1024}}}{}
\pysigstopsignatures
\sphinxAtStartPar
Advanced implementation that tiles both database and query points.
This wrapper handles the dynamic aspects before calling the JIT\sphinxhyphen{}compiled
function.
\begin{quote}\begin{description}
\sphinxlineitem{Parameters}\begin{itemize}
\item {} 
\sphinxAtStartPar
\sphinxstyleliteralstrong{\sphinxupquote{x}} \textendash{} (n, d) array of database points

\item {} 
\sphinxAtStartPar
\sphinxstyleliteralstrong{\sphinxupquote{y}} \textendash{} (m, d) array of query points

\item {} 
\sphinxAtStartPar
\sphinxstyleliteralstrong{\sphinxupquote{k}} \textendash{} number of nearest neighbors

\item {} 
\sphinxAtStartPar
\sphinxstyleliteralstrong{\sphinxupquote{x\_tile\_size}} \textendash{} size of database tiles

\item {} 
\sphinxAtStartPar
\sphinxstyleliteralstrong{\sphinxupquote{y\_batch\_size}} \textendash{} size of query batches

\end{itemize}

\sphinxlineitem{Returns}
\sphinxAtStartPar
(m, k) array of indices of nearest neighbors

\end{description}\end{quote}

\end{fulllineitems}


\end{fulllineitems}



\subsection{graphem.influence module}
\label{\detokenize{graphem:module-graphem.influence}}\label{\detokenize{graphem:graphem-influence-module}}\index{module@\spxentry{module}!graphem.influence@\spxentry{graphem.influence}}\index{graphem.influence@\spxentry{graphem.influence}!module@\spxentry{module}}
\sphinxAtStartPar
Influence maximization functionality for Graphem.
\index{graphem\_seed\_selection() (in module graphem.influence)@\spxentry{graphem\_seed\_selection()}\spxextra{in module graphem.influence}}

\begin{fulllineitems}
\phantomsection\label{\detokenize{graphem:graphem.influence.graphem_seed_selection}}
\pysigstartsignatures
\pysiglinewithargsret{\sphinxcode{\sphinxupquote{graphem.influence.}}\sphinxbfcode{\sphinxupquote{graphem\_seed\_selection}}}{\sphinxparam{\DUrole{n,n}{embedder}}\sphinxparamcomma \sphinxparam{\DUrole{n,n}{k}}\sphinxparamcomma \sphinxparam{\DUrole{n,n}{num\_iterations}\DUrole{o,o}{=}\DUrole{default_value}{20}}}{}
\pysigstopsignatures
\sphinxAtStartPar
Run the GraphEmbedder layout to get an embedding, then select
k seeds by choosing the nodes with the highest radial distances.
\begin{quote}\begin{description}
\sphinxlineitem{Parameters}\begin{itemize}
\item {} 
\sphinxAtStartPar
\sphinxstyleliteralstrong{\sphinxupquote{embedder}} \textendash{} GraphEmbedder
The initialized graph embedder object

\item {} 
\sphinxAtStartPar
\sphinxstyleliteralstrong{\sphinxupquote{k}} \textendash{} int
Number of seed nodes to select

\item {} 
\sphinxAtStartPar
\sphinxstyleliteralstrong{\sphinxupquote{num\_iterations}} \textendash{} int
Number of layout iterations to run

\end{itemize}

\sphinxlineitem{Returns}
\sphinxAtStartPar
\begin{description}
\sphinxlineitem{list}
\sphinxAtStartPar
List of k vertices selected as seeds

\end{description}


\sphinxlineitem{Return type}
\sphinxAtStartPar
seeds

\end{description}\end{quote}

\end{fulllineitems}

\index{ndlib\_estimated\_influence() (in module graphem.influence)@\spxentry{ndlib\_estimated\_influence()}\spxextra{in module graphem.influence}}

\begin{fulllineitems}
\phantomsection\label{\detokenize{graphem:graphem.influence.ndlib_estimated_influence}}
\pysigstartsignatures
\pysiglinewithargsret{\sphinxcode{\sphinxupquote{graphem.influence.}}\sphinxbfcode{\sphinxupquote{ndlib\_estimated\_influence}}}{\sphinxparam{\DUrole{n,n}{G}}\sphinxparamcomma \sphinxparam{\DUrole{n,n}{seeds}}\sphinxparamcomma \sphinxparam{\DUrole{n,n}{p}\DUrole{o,o}{=}\DUrole{default_value}{0.1}}\sphinxparamcomma \sphinxparam{\DUrole{n,n}{iterations\_count}\DUrole{o,o}{=}\DUrole{default_value}{200}}}{}
\pysigstopsignatures
\sphinxAtStartPar
Run NDlib’s Independent Cascades model on graph G, starting with the given seeds,
and return the estimated final influence (number of nodes in state 2) and
the number of iterations executed.
\begin{quote}\begin{description}
\sphinxlineitem{Parameters}\begin{itemize}
\item {} 
\sphinxAtStartPar
\sphinxstyleliteralstrong{\sphinxupquote{G}} \textendash{} networkx.Graph
The graph to run influence propagation on

\item {} 
\sphinxAtStartPar
\sphinxstyleliteralstrong{\sphinxupquote{seeds}} \textendash{} list
The list of seed nodes

\item {} 
\sphinxAtStartPar
\sphinxstyleliteralstrong{\sphinxupquote{p}} \textendash{} float
Propagation probability

\item {} 
\sphinxAtStartPar
\sphinxstyleliteralstrong{\sphinxupquote{iterations\_count}} \textendash{} int
Maximum number of simulation iterations

\end{itemize}

\sphinxlineitem{Returns}
\sphinxAtStartPar
\begin{description}
\sphinxlineitem{float}
\sphinxAtStartPar
The estimated influence (average number of influenced nodes)

\sphinxlineitem{iterations: int}
\sphinxAtStartPar
The number of iterations run

\end{description}


\sphinxlineitem{Return type}
\sphinxAtStartPar
influence

\end{description}\end{quote}

\end{fulllineitems}

\index{greedy\_seed\_selection() (in module graphem.influence)@\spxentry{greedy\_seed\_selection()}\spxextra{in module graphem.influence}}

\begin{fulllineitems}
\phantomsection\label{\detokenize{graphem:graphem.influence.greedy_seed_selection}}
\pysigstartsignatures
\pysiglinewithargsret{\sphinxcode{\sphinxupquote{graphem.influence.}}\sphinxbfcode{\sphinxupquote{greedy\_seed\_selection}}}{\sphinxparam{\DUrole{n,n}{G}}\sphinxparamcomma \sphinxparam{\DUrole{n,n}{k}}\sphinxparamcomma \sphinxparam{\DUrole{n,n}{p}\DUrole{o,o}{=}\DUrole{default_value}{0.1}}\sphinxparamcomma \sphinxparam{\DUrole{n,n}{iterations\_count}\DUrole{o,o}{=}\DUrole{default_value}{200}}}{}
\pysigstopsignatures
\sphinxAtStartPar
Greedy seed selection using NDlib influence estimation.
For each candidate node evaluation, it calls NDlib’s simulation and accumulates
the total number of iterations used across all evaluations.
\begin{quote}\begin{description}
\sphinxlineitem{Returns}
\sphinxAtStartPar
the selected seed set (list of nodes)
total\_iters: the total number of NDlib iterations run during selection.

\sphinxlineitem{Return type}
\sphinxAtStartPar
seeds

\end{description}\end{quote}

\end{fulllineitems}



\subsection{graphem.visualization module}
\label{\detokenize{graphem:module-graphem.visualization}}\label{\detokenize{graphem:graphem-visualization-module}}\index{module@\spxentry{module}!graphem.visualization@\spxentry{graphem.visualization}}\index{graphem.visualization@\spxentry{graphem.visualization}!module@\spxentry{module}}
\sphinxAtStartPar
Visualization utilities for Graphem.
\index{report\_corr() (in module graphem.visualization)@\spxentry{report\_corr()}\spxextra{in module graphem.visualization}}

\begin{fulllineitems}
\phantomsection\label{\detokenize{graphem:graphem.visualization.report_corr}}
\pysigstartsignatures
\pysiglinewithargsret{\sphinxcode{\sphinxupquote{graphem.visualization.}}\sphinxbfcode{\sphinxupquote{report\_corr}}}{\sphinxparam{\DUrole{n,n}{name}}\sphinxparamcomma \sphinxparam{\DUrole{n,n}{radii}}\sphinxparamcomma \sphinxparam{\DUrole{n,n}{centrality}}\sphinxparamcomma \sphinxparam{\DUrole{n,n}{alpha}\DUrole{o,o}{=}\DUrole{default_value}{0.025}}}{}
\pysigstopsignatures
\sphinxAtStartPar
Calculate and report the correlation between radial distances and a centrality measure.
\begin{quote}\begin{description}
\sphinxlineitem{Parameters}\begin{itemize}
\item {} 
\sphinxAtStartPar
\sphinxstyleliteralstrong{\sphinxupquote{name}} \textendash{} str
Name of the centrality measure

\item {} 
\sphinxAtStartPar
\sphinxstyleliteralstrong{\sphinxupquote{radii}} \textendash{} array\sphinxhyphen{}like
Radial distances from origin

\item {} 
\sphinxAtStartPar
\sphinxstyleliteralstrong{\sphinxupquote{centrality}} \textendash{} array\sphinxhyphen{}like
Centrality values

\item {} 
\sphinxAtStartPar
\sphinxstyleliteralstrong{\sphinxupquote{alpha}} \textendash{} float
Alpha level for confidence interval

\end{itemize}

\sphinxlineitem{Returns}
\sphinxAtStartPar
(correlation coefficient, p\sphinxhyphen{}value)

\sphinxlineitem{Return type}
\sphinxAtStartPar
\sphinxhref{https://docs.python.org/3/library/stdtypes.html\#tuple}{tuple}

\end{description}\end{quote}

\end{fulllineitems}

\index{report\_full\_correlation\_matrix() (in module graphem.visualization)@\spxentry{report\_full\_correlation\_matrix()}\spxextra{in module graphem.visualization}}

\begin{fulllineitems}
\phantomsection\label{\detokenize{graphem:graphem.visualization.report_full_correlation_matrix}}
\pysigstartsignatures
\pysiglinewithargsret{\sphinxcode{\sphinxupquote{graphem.visualization.}}\sphinxbfcode{\sphinxupquote{report\_full\_correlation\_matrix}}}{\sphinxparam{\DUrole{n,n}{radii}}\sphinxparamcomma \sphinxparam{\DUrole{n,n}{deg}}\sphinxparamcomma \sphinxparam{\DUrole{n,n}{btw}}\sphinxparamcomma \sphinxparam{\DUrole{n,n}{eig}}\sphinxparamcomma \sphinxparam{\DUrole{n,n}{pr}}\sphinxparamcomma \sphinxparam{\DUrole{n,n}{clo}}\sphinxparamcomma \sphinxparam{\DUrole{n,n}{nload}}\sphinxparamcomma \sphinxparam{\DUrole{n,n}{alpha}\DUrole{o,o}{=}\DUrole{default_value}{0.025}}}{}
\pysigstopsignatures
\sphinxAtStartPar
Calculate and report correlations between radial distances and various centrality measures.
\begin{quote}\begin{description}
\sphinxlineitem{Parameters}\begin{itemize}
\item {} 
\sphinxAtStartPar
\sphinxstyleliteralstrong{\sphinxupquote{radii}} \textendash{} array\sphinxhyphen{}like
Radial distances from origin

\item {} 
\sphinxAtStartPar
\sphinxstyleliteralstrong{\sphinxupquote{deg}} \textendash{} array\sphinxhyphen{}like
Various centrality measures

\item {} 
\sphinxAtStartPar
\sphinxstyleliteralstrong{\sphinxupquote{btw}} \textendash{} array\sphinxhyphen{}like
Various centrality measures

\item {} 
\sphinxAtStartPar
\sphinxstyleliteralstrong{\sphinxupquote{eig}} \textendash{} array\sphinxhyphen{}like
Various centrality measures

\item {} 
\sphinxAtStartPar
\sphinxstyleliteralstrong{\sphinxupquote{pr}} \textendash{} array\sphinxhyphen{}like
Various centrality measures

\item {} 
\sphinxAtStartPar
\sphinxstyleliteralstrong{\sphinxupquote{clo}} \textendash{} array\sphinxhyphen{}like
Various centrality measures

\item {} 
\sphinxAtStartPar
\sphinxstyleliteralstrong{\sphinxupquote{edge\_btw}} \textendash{} array\sphinxhyphen{}like
Various centrality measures

\item {} 
\sphinxAtStartPar
\sphinxstyleliteralstrong{\sphinxupquote{alpha}} \textendash{} float
Alpha level for confidence interval

\end{itemize}

\sphinxlineitem{Returns}
\sphinxAtStartPar
Correlation matrix

\sphinxlineitem{Return type}
\sphinxAtStartPar
pandas.DataFrame

\end{description}\end{quote}

\end{fulllineitems}

\index{plot\_radial\_vs\_centrality() (in module graphem.visualization)@\spxentry{plot\_radial\_vs\_centrality()}\spxextra{in module graphem.visualization}}

\begin{fulllineitems}
\phantomsection\label{\detokenize{graphem:graphem.visualization.plot_radial_vs_centrality}}
\pysigstartsignatures
\pysiglinewithargsret{\sphinxcode{\sphinxupquote{graphem.visualization.}}\sphinxbfcode{\sphinxupquote{plot\_radial\_vs\_centrality}}}{\sphinxparam{\DUrole{n,n}{radii}}\sphinxparamcomma \sphinxparam{\DUrole{n,n}{centralities}}\sphinxparamcomma \sphinxparam{\DUrole{n,n}{names}}}{}
\pysigstopsignatures
\sphinxAtStartPar
Plot scatter plots of radial distances vs. various centrality measures.
\begin{quote}\begin{description}
\sphinxlineitem{Parameters}\begin{itemize}
\item {} 
\sphinxAtStartPar
\sphinxstyleliteralstrong{\sphinxupquote{radii}} \textendash{} array\sphinxhyphen{}like
Radial distances from origin

\item {} 
\sphinxAtStartPar
\sphinxstyleliteralstrong{\sphinxupquote{centralities}} \textendash{} list of array\sphinxhyphen{}like
List of centrality measures

\item {} 
\sphinxAtStartPar
\sphinxstyleliteralstrong{\sphinxupquote{names}} \textendash{} list of str
Names of the centrality measures

\end{itemize}

\end{description}\end{quote}

\end{fulllineitems}

\index{display\_benchmark\_results() (in module graphem.visualization)@\spxentry{display\_benchmark\_results()}\spxextra{in module graphem.visualization}}

\begin{fulllineitems}
\phantomsection\label{\detokenize{graphem:graphem.visualization.display_benchmark_results}}
\pysigstartsignatures
\pysiglinewithargsret{\sphinxcode{\sphinxupquote{graphem.visualization.}}\sphinxbfcode{\sphinxupquote{display\_benchmark\_results}}}{\sphinxparam{\DUrole{n,n}{benchmark\_results}}}{}
\pysigstopsignatures
\sphinxAtStartPar
Display benchmark results in a nicely formatted table.
\begin{quote}\begin{description}
\sphinxlineitem{Parameters}
\sphinxAtStartPar
\sphinxstyleliteralstrong{\sphinxupquote{benchmark\_results}} \textendash{} list of dict
List of benchmark result dictionaries

\end{description}\end{quote}

\end{fulllineitems}



\subsection{Module contents}
\label{\detokenize{graphem:module-graphem}}\label{\detokenize{graphem:module-contents}}\index{module@\spxentry{module}!graphem@\spxentry{graphem}}\index{graphem@\spxentry{graphem}!module@\spxentry{module}}
\sphinxAtStartPar
Graphem: A graph embedding library based on JAX for efficient k\sphinxhyphen{}nearest neighbors.

\sphinxAtStartPar
GraphEm is a graph embedding library based on JAX for efficient centrality measures approximation and influence maximization in networks.


\chapter{Overview}
\label{\detokenize{index:overview}}
\sphinxAtStartPar
Graphem is a Python library for graph visualization and analysis, with a focus on efficient embedding of large networks. It uses JAX for accelerated computation and provides tools for influence maximization in networks.

\sphinxAtStartPar
Key features:
\begin{itemize}
\item {} 
\sphinxAtStartPar
Fast graph embedding using Laplacian embedding

\item {} 
\sphinxAtStartPar
Efficient k\sphinxhyphen{}nearest neighbors search with JAX

\item {} 
\sphinxAtStartPar
Various graph generation models

\item {} 
\sphinxAtStartPar
Tools for influence maximization

\item {} 
\sphinxAtStartPar
Graph visualization with Plotly

\item {} 
\sphinxAtStartPar
Benchmarking tools for comparing graph metrics

\end{itemize}


\chapter{Installation}
\label{\detokenize{index:installation}}
\sphinxAtStartPar
To install from PyPI:

\begin{sphinxVerbatim}[commandchars=\\\{\}]
\PYG{n}{pip} \PYG{n}{install} \PYG{n}{graphem}\PYG{o}{\PYGZhy{}}\PYG{n}{jax}
\end{sphinxVerbatim}

\sphinxAtStartPar
To install from the GitHub repository:

\begin{sphinxVerbatim}[commandchars=\\\{\}]
\PYG{n}{pip} \PYG{n}{install} \PYG{n}{git}\PYG{o}{+}\PYG{n}{https}\PYG{p}{:}\PYG{o}{/}\PYG{o}{/}\PYG{n}{github}\PYG{o}{.}\PYG{n}{com}\PYG{o}{/}\PYG{n}{sashakolpakov}\PYG{o}{/}\PYG{n}{graphem}\PYG{o}{.}\PYG{n}{git}
\end{sphinxVerbatim}


\chapter{Quick Start}
\label{\detokenize{index:quick-start}}

\section{Basic Graph Embedding}
\label{\detokenize{index:basic-graph-embedding}}
\begin{sphinxVerbatim}[commandchars=\\\{\}]
\PYG{k+kn}{from}\PYG{+w}{ }\PYG{n+nn}{graphem}\PYG{n+nn}{.}\PYG{n+nn}{generators}\PYG{+w}{ }\PYG{k+kn}{import} \PYG{n}{erdos\PYGZus{}renyi\PYGZus{}graph}
\PYG{k+kn}{from}\PYG{+w}{ }\PYG{n+nn}{graphem}\PYG{n+nn}{.}\PYG{n+nn}{embedder}\PYG{+w}{ }\PYG{k+kn}{import} \PYG{n}{GraphEmbedder}

\PYG{c+c1}{\PYGZsh{} Generate a random graph}
\PYG{n}{n\PYGZus{}vertices} \PYG{o}{=} \PYG{l+m+mi}{200}
\PYG{n}{edges} \PYG{o}{=} \PYG{n}{erdos\PYGZus{}renyi\PYGZus{}graph}\PYG{p}{(}\PYG{n}{n}\PYG{o}{=}\PYG{n}{n\PYGZus{}vertices}\PYG{p}{,} \PYG{n}{p}\PYG{o}{=}\PYG{l+m+mf}{0.05}\PYG{p}{)}

\PYG{c+c1}{\PYGZsh{} Create an embedder}
\PYG{n}{embedder} \PYG{o}{=} \PYG{n}{GraphEmbedder}\PYG{p}{(}
    \PYG{n}{edges}\PYG{o}{=}\PYG{n}{edges}\PYG{p}{,}
    \PYG{n}{n\PYGZus{}vertices}\PYG{o}{=}\PYG{n}{n\PYGZus{}vertices}\PYG{p}{,}
    \PYG{n}{dimension}\PYG{o}{=}\PYG{l+m+mi}{3}\PYG{p}{,}  \PYG{c+c1}{\PYGZsh{} 3D embedding}
    \PYG{n}{L\PYGZus{}min}\PYG{o}{=}\PYG{l+m+mf}{10.0}\PYG{p}{,}   \PYG{c+c1}{\PYGZsh{} Minimum edge length}
    \PYG{n}{k\PYGZus{}attr}\PYG{o}{=}\PYG{l+m+mf}{0.5}\PYG{p}{,}   \PYG{c+c1}{\PYGZsh{} Attraction force constant}
    \PYG{n}{k\PYGZus{}inter}\PYG{o}{=}\PYG{l+m+mf}{0.1}\PYG{p}{,}  \PYG{c+c1}{\PYGZsh{} Repulsion force constant}
    \PYG{n}{knn\PYGZus{}k}\PYG{o}{=}\PYG{l+m+mi}{15}      \PYG{c+c1}{\PYGZsh{} Number of nearest neighbors}
\PYG{p}{)}

\PYG{c+c1}{\PYGZsh{} Run the layout algorithm}
\PYG{n}{embedder}\PYG{o}{.}\PYG{n}{run\PYGZus{}layout}\PYG{p}{(}\PYG{n}{num\PYGZus{}iterations}\PYG{o}{=}\PYG{l+m+mi}{40}\PYG{p}{)}

\PYG{c+c1}{\PYGZsh{} Visualize the graph}
\PYG{n}{embedder}\PYG{o}{.}\PYG{n}{display\PYGZus{}layout}\PYG{p}{(}\PYG{n}{edge\PYGZus{}width}\PYG{o}{=}\PYG{l+m+mf}{0.5}\PYG{p}{,} \PYG{n}{node\PYGZus{}size}\PYG{o}{=}\PYG{l+m+mi}{5}\PYG{p}{)}
\end{sphinxVerbatim}


\section{Influence Maximization}
\label{\detokenize{index:influence-maximization}}
\begin{sphinxVerbatim}[commandchars=\\\{\}]
\PYG{k+kn}{import}\PYG{+w}{ }\PYG{n+nn}{networkx}\PYG{+w}{ }\PYG{k}{as}\PYG{+w}{ }\PYG{n+nn}{nx}
\PYG{k+kn}{from}\PYG{+w}{ }\PYG{n+nn}{graphem}\PYG{n+nn}{.}\PYG{n+nn}{influence}\PYG{+w}{ }\PYG{k+kn}{import} \PYG{n}{graphem\PYGZus{}seed\PYGZus{}selection}\PYG{p}{,} \PYG{n}{ndlib\PYGZus{}estimated\PYGZus{}influence}

\PYG{c+c1}{\PYGZsh{} Convert edges to NetworkX graph}
\PYG{n}{G} \PYG{o}{=} \PYG{n}{nx}\PYG{o}{.}\PYG{n}{Graph}\PYG{p}{(}\PYG{p}{)}
\PYG{n}{G}\PYG{o}{.}\PYG{n}{add\PYGZus{}nodes\PYGZus{}from}\PYG{p}{(}\PYG{n+nb}{range}\PYG{p}{(}\PYG{n}{n\PYGZus{}vertices}\PYG{p}{)}\PYG{p}{)}
\PYG{n}{G}\PYG{o}{.}\PYG{n}{add\PYGZus{}edges\PYGZus{}from}\PYG{p}{(}\PYG{n}{edges}\PYG{p}{)}

\PYG{c+c1}{\PYGZsh{} Select seed nodes using the Graphem method}
\PYG{n}{seeds} \PYG{o}{=} \PYG{n}{graphem\PYGZus{}seed\PYGZus{}selection}\PYG{p}{(}\PYG{n}{embedder}\PYG{p}{,} \PYG{n}{k}\PYG{o}{=}\PYG{l+m+mi}{10}\PYG{p}{,} \PYG{n}{num\PYGZus{}iterations}\PYG{o}{=}\PYG{l+m+mi}{20}\PYG{p}{)}

\PYG{c+c1}{\PYGZsh{} Estimate influence}
\PYG{n}{influence}\PYG{p}{,} \PYG{n}{iterations} \PYG{o}{=} \PYG{n}{ndlib\PYGZus{}estimated\PYGZus{}influence}\PYG{p}{(}\PYG{n}{G}\PYG{p}{,} \PYG{n}{seeds}\PYG{p}{,} \PYG{n}{p}\PYG{o}{=}\PYG{l+m+mf}{0.1}\PYG{p}{,} \PYG{n}{iterations\PYGZus{}count}\PYG{o}{=}\PYG{l+m+mi}{200}\PYG{p}{)}
\PYG{n+nb}{print}\PYG{p}{(}\PYG{l+s+sa}{f}\PYG{l+s+s2}{\PYGZdq{}}\PYG{l+s+s2}{Estimated influence: }\PYG{l+s+si}{\PYGZob{}}\PYG{n}{influence}\PYG{l+s+si}{\PYGZcb{}}\PYG{l+s+s2}{ nodes (}\PYG{l+s+si}{\PYGZob{}}\PYG{n}{influence}\PYG{o}{/}\PYG{n}{n\PYGZus{}vertices}\PYG{l+s+si}{:}\PYG{l+s+s2}{.2\PYGZpc{}}\PYG{l+s+si}{\PYGZcb{}}\PYG{l+s+s2}{ of the graph)}\PYG{l+s+s2}{\PYGZdq{}}\PYG{p}{)}
\end{sphinxVerbatim}


\chapter{API Reference}
\label{\detokenize{index:module-graphem}}\label{\detokenize{index:api-reference}}\index{module@\spxentry{module}!graphem@\spxentry{graphem}}\index{graphem@\spxentry{graphem}!module@\spxentry{module}}
\sphinxAtStartPar
Graphem: A graph embedding library based on JAX for efficient k\sphinxhyphen{}nearest neighbors.


\chapter{Indices and tables}
\label{\detokenize{index:indices-and-tables}}\begin{itemize}
\item {} 
\sphinxAtStartPar
\DUrole{xref,std,std-ref}{genindex}

\item {} 
\sphinxAtStartPar
\DUrole{xref,std,std-ref}{modindex}

\item {} 
\sphinxAtStartPar
\DUrole{xref,std,std-ref}{search}

\end{itemize}


\renewcommand{\indexname}{Python Module Index}
\begin{sphinxtheindex}
\let\bigletter\sphinxstyleindexlettergroup
\bigletter{g}
\item\relax\sphinxstyleindexentry{graphem}\sphinxstyleindexpageref{index:\detokenize{module-graphem}}
\item\relax\sphinxstyleindexentry{graphem.benchmark}\sphinxstyleindexpageref{graphem:\detokenize{module-graphem.benchmark}}
\item\relax\sphinxstyleindexentry{graphem.datasets}\sphinxstyleindexpageref{graphem:\detokenize{module-graphem.datasets}}
\item\relax\sphinxstyleindexentry{graphem.embedder}\sphinxstyleindexpageref{graphem:\detokenize{module-graphem.embedder}}
\item\relax\sphinxstyleindexentry{graphem.generators}\sphinxstyleindexpageref{graphem:\detokenize{module-graphem.generators}}
\item\relax\sphinxstyleindexentry{graphem.index}\sphinxstyleindexpageref{graphem:\detokenize{module-graphem.index}}
\item\relax\sphinxstyleindexentry{graphem.influence}\sphinxstyleindexpageref{graphem:\detokenize{module-graphem.influence}}
\item\relax\sphinxstyleindexentry{graphem.visualization}\sphinxstyleindexpageref{graphem:\detokenize{module-graphem.visualization}}
\end{sphinxtheindex}

\renewcommand{\indexname}{Index}
\printindex
\end{document}